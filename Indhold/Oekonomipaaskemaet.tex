Man hører oftere og oftere i diverse medier, at unge mennesker i stigende grad har svært ved at administrere deres egen økonomi. Dette udmønter sig blandt andet i artikler og tv-programmer omkring emnet. Især tv-programmer som Luksusfælden, der netop har fokus på familiers manglende kontrol af deres økonomi, skaber opmærksomhed på problemstillingen. Denne stigende interesse for emnet er dog heller ikke gået politikerne hen over hovedet.

Politikerne har gennem tiden indført forskellige redskaber lærerne rundt om i landet kan bruge i undervisningen om privatøkonomi. Det er dog ikke tilstrækkeligt, i hvert fald ikke hvis man spørger bankerne. Bankernes brancheorganisation udtalte i 2011, at de unge ikke har tilstrækkeligt styr på håndteringen af privatøkonomi, og slet ikke i en krisetid som den Danmark befandt sig i på daværende tidspunkt. De opfordrede derfor politikerne til at opruste undervisningen. En sådan udtalelse bygger blandt andet på en undersøgelse Danske Bank lavede i år 2010((((REF)))), hvor de opdagede at 37\% af de 18 til 27-årige ikke præcist ved hvad ordet ”rente” dækker over. !!!!!Det har ikke været muligt at finde ud af om der er blevet lavet konkrete lovændringer, dog har vi undersøgt reglerne som de er i dag. I matematik skal de unge kunne anvende blandt andet procentberegninger og formler, og de kommer i faget til at arbejde med almindelig indkøb, transport, boligforhold og lønopgørelser bare for at nævne nogle få. Det er også værd at understrege at finansielle forhold og økonomi indgår i folkeskolen afgangsprøve. I samfundsfag er der også allerede i dag taget højde for unges manglende forståelse af økonomi. Her er det et blandt andet et krav at de unge lærer om fagforeninger, budgetter, lån og skat. Det er ting som disse, der har fået nogle politikere til at mene der ikke er noget i vejen med lovgivningen, som den er nu, men problemet ligger i at kommunerne, og i sidste ende skolerne, ikke lever op til loven.

Liste over hvad man skal igennem i folkeskolen vedrørende privatøkonomi.
Problemstillinger man skal arbejde med i matematik i folkeskolen:
Behandle eksempler på problemstillinger knyttet til den samfundsmæssige udvikling, hvori bl.a. økonomi, teknologi og miljø indgår
Anvende faglige redskaber og begreber, bl.a. procentberegninger, formler og funktioner som værktøj til løsning af praktiske problemer
Udføre simuleringer, blandt andet ved hjælp af it
Erkende matematikken muligheder og begrænsninger som beskrivelsesmiddel og beslutningsgrundlag.

\begin{itemize}
\item{Læseplan:}
\subitem{Dagligdagens indkøb, transport og boligforhold}
\subitem{Lønopgørelser og skatteberegninger}
\subitem{Rentebegrebet, fx i tilknytning til opsparing, låntagning og kreditkøb}
\subitem{Miljø, teknologi og levevilkår, fx energiforbrug, affald og ressourcer}
\subitem{Simuleringer og statistiske beskrivelser}
\subitem{Desuden indgår emner vedrørende finansielle forhold og økonomi i dag i afgangsprøverne i matematik}


\item{Privatøkonomi i samfundsfag i folkeskolen}
\subitem{Det fremgår af Fælles Mål for faget samfundsfag, at eleverne blandt andet skal:}
\subitem{Redegøre for hovedtræk i udviklingen i dansk erhvervs- og produktionsstruktur, herunder centrale aktører på arbejdsmarkedet og deres interesser}
\subitem{Redegøre for det økonomiske kredsløb og markedsmekanismer}
\subitem{Redegøre for centrale velfærdsprincipper og typer af velfærdsstater}
\subitem{Forstå og forklare udsagn om økonomi set i forhold til forskellige aktørers interesser og ideologier}
\subitem{Reflektere over økonomiens betydning for det danske velfærdssamfund}
\subitem{Diskutere mulige handlinger i relation til virkninger af økonomiens globalisering}
\item{Alle disse mål kan udmøntes med emner, hvor fagets redskaber skal bruges. Det kan for eksempel være:}
\subitem{Fagforeninger, a-kasser}
\subitem{Budget, lån, lønseddel, feriepenge}
\subitem{Budget, bolig, forsikring}
\subitem{Skat, fradrag, opsparing}
\subitem{Pension og fradrag}
\subitem{SU, lån, husholdning}
\subitem{Ved den mundtlige lokalt stillede prøve i samfundsfag prøves eleverne inden for de faglige mål, herunder økonomi og finansiering}
\end{itemize}

Dog har mange kommuner og enkeltstående skoler har selv taget initiativ til at gøre en ekstra indsats mod dette problem, Dette sker på blandt andet temadage og lignende. På Sindal-skole har de gjort netop dette. De har brugt det føromtalte program, Luksusfælden, i undervisningen, og taget på besøg hos lokale banker for at få råd og vejledning af folk der arbejder med emnet på daglig basis.
Man kan tolke denne udvikling på flere måder, men det er svært at komme udenom at der må være brug for initiativer som dette, hvilket vil sige at der muligvis også er brug for vores projekt.


 