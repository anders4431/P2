Man hører oftere og oftere i diverse medier, at unge mennesker i stigende grad har svært ved at administrere deres egen økonomi. Dette udmønter sig blandt andet i artikler og tv-programmer omkring emnet. Især tv-programmer som Luksusfælden, der netop har fokus på familiers manglende kontrol af deres økonomi, skaber opmærksomhed på problemstillingen. Denne stigende interesse for emnet er heller ikke gået politikerne hen over hovedet. Især i 2010/11 var der stor politisk interesse for emnet, denne er dog ebbet lidt ud sidenhen.

Politikerne har gennem tiden indført forskellige redskaber lærerne rundt om i landet kan bruge i undervisningen om privatøkonomi. Det er dog ikke tilstrækkeligt, i hvert fald ikke hvis man spørger bankerne. Bankernes brancheorganisation udtalte i 2011, at de unge ikke har tilstrækkeligt styr på håndteringen af privatøkonomi, og slet ikke i en krisetid som den Danmark befandt sig i på daværende tidspunkt. De opfordrede derfor politikerne til at opruste undervisningen. En sådan udtalelse bygger blandt andet på samme  undersøgelse som nævnt i afsnit \ref{UvidendeUnge}. Undersøgelsen hvor man opdagede at 37\% af de 18 til 27-årige ikke præcist ved hvad ordet ”rente” dækker over. [[!!!!!]] Det har ikke været muligt at finde ud af om der er blevet lavet konkrete lovændringer, dog har vi undersøgt reglerne som de er i dag. 

[[VURDÉR OM DER MANGLER NOGET PÅ LISTEN]]
\begin{quotation} \itshape
Liste over hvad man skal igennem i folkeskolen vedrørende privatøkonomi.
Problemstillinger man skal arbejde med i matematik i folkeskolen:
Behandle eksempler på problemstillinger knyttet til den samfundsmæssige udvikling, hvori bl.a. økonomi, teknologi og miljø indgår
Anvende faglige redskaber og begreber, bl.a. procentberegninger, formler og funktioner som værktøj til løsning af praktiske problemer
Udføre simuleringer, blandt andet ved hjælp af it
Erkende matematikkens muligheder og begrænsninger som beskrivelsesmiddel og beslutningsgrundlag.

\begin{description}
\item[{Læseplan:}] \hfill
\noindent \begin{itemize}
\item{Dagligdagens indkøb, transport og boligforhold}
\item{Lønopgørelser og skatteberegninger}
\item{Rentebegrebet, fx i tilknytning til opsparing, låntagning og kreditkøb}
\item{Miljø, teknologi og levevilkår, fx energiforbrug, affald og ressourcer}
\item{Simuleringer og statistiske beskrivelser}
\item{Desuden indgår emner vedrørende finansielle forhold og økonomi i dag i afgangsprøverne i matematik}
\cite{FallesMalMatematik}
\end{itemize}

\item[Privatøkonomi i samfundsfag i folkeskolen] \hfill
\noindent \begin{itemize}
\item{Det fremgår af Fælles Mål for faget samfundsfag, at eleverne blandt andet skal:}
\item{Redegøre for hovedtræk i udviklingen i dansk erhvervs- og produktionsstruktur, herunder centrale aktører på arbejdsmarkedet og deres interesser}
\item{Redegøre for det økonomiske kredsløb og markedsmekanismer}
\item{Redegøre for centrale velfærdsprincipper og typer af velfærdsstater}
\item{Forstå og forklare udsagn om økonomi set i forhold til forskellige aktørers interesser og ideologier}
\item{Reflektere over økonomiens betydning for det danske velfærdssamfund}
\item{Diskutere mulige handlinger i relation til virkninger af økonomiens globalisering}
\end{itemize}

\item{Alle disse mål kan udmøntes med emner, hvor fagets redskaber skal bruges. Det kan for eksempel være:}
\noindent \begin{itemize}
\item{Fagforeninger, a-kasser}
\item{Budget, lån, lønseddel, feriepenge}
\item{Budget, bolig, forsikring}
\item{Skat, fradrag, opsparing}
\item{Pension og fradrag}
\item{SU, lån, husholdning}
\item{Ved den mundtlige lokalt stillede prøve i samfundsfag prøves eleverne inden for de faglige mål, herunder økonomi og finansiering}
\cite{FallesMalSamfundsfag}
\end{itemize}
\end{description}
\end{quotation}

Vi har vurderet at listen dækker alle de basale og nødvendige begreber unge mennesker har brug for, for at kunne klare sig selv og styre sin egen økonomi, når de kommer ud på egne ben. Det er altså ikke fordi de unge ikke modtager undervisning på dette område, at de ikke er gode til at styre deres egen økonomi. Det er dog muligt at nogle af termerne er underprioriteret i undervisningen.

I matematik skal de unge altså kunne anvende blandt andet procentberegninger og formler, og de kommer i faget til at arbejde med almindelig indkøb, transport, boligforhold og lønopgørelser. Det er også værd at understrege at finansielle forhold og økonomi indgår i folkeskolen afgangsprøve. Det mest iøjnefaldende må dog være at de unge rent faktisk relativt tidligt, skal lære om rentebegrebet, hvilket førnævnte undersøgelse viste at en stor del af de unge faktisk ikke har opnået. Grunden til dette er muligvis at der bliver stillet for høje krav i en for tidlig alder. Det er da også muligt at undervisningen på dette område bare ikke er tilstrækkelig.
I samfundsfag er der også allerede i dag taget højde for unges manglende forståelse af økonomi. Her er det et blandt andet et krav at de unge lærer om fagforeninger, budgetter, lån og skat. Det er ting som disse, der har fået nogle politikere til at mene der ikke er noget i vejen med lovgivningen på området, som den er nu, men problemet ligger i at kommunerne, og i sidste ende skolerne, ikke lever op til loven\cite{BusinessDK3}. Man kan så yderligere spørge sig selv om hvorvidt forældrene lever op til deres ansvar. De senere år er diskussionen, om hvorvidt forældrene lever op til deres ansvar kontra lægger mere og mere ansvar over på skolerne, blusset kraftigt op. Det er ikke sikkert at det kun er skolerne/lovgivningen der er noget galt med, måske forældrene mangler at tage ansvar hvilket hypotetisk set kunne have påvirket de unge i en positiv retning. Forældreansvaret kan forstås på to måder, den ene er at forældrene burde sørge for at deres barn får lavet sine lektier, hvilket vil gøre at barnet er bedre forberedt og potientielt bedre stillet til at lære om blandt andet økonomi. Den anden måde det kan forstås på er, at forældrene bør ruste barnet til det fremtidige liv. Dette inkluderer naturligvis også hvordan man håndterer sin egen økonomi. Grunden til dette kan være mange, man kan overveje om det er fordi forældrene har for travlt, ikke selv er gode til at håndtere økonomien eller måske noget helt tredje. Denne holdningsændring har blandt andet medført forslag om at lærerne skulle stå til rådighed selv om aftenen når eleven sidder med lektierne. En ret så radikalt forslag havde man nok ikke forestillet for bare 10 år siden\cite{ForaldreAnsvar}.

Dog har flere kommuner og enkeltstående skoler har selv taget initiativ til at gøre en ekstra indsats mod dette problem, Dette sker på blandt andet temadage og lignende. På Sindal-skole har de gjort netop dette. De har brugt det føromtalte program, Luksusfælden, i undervisningen, og taget på besøg hos lokale banker for at få råd og vejledning af folk der arbejder med emnet på daglig basis \cite{Sindal}. Man kan tolke denne udvikling på flere måder, men det er svært at komme udenom at der må være brug for initiativer som dette, hvilket vil sige at der muligvis også er brug for vores projekt.






 