Udfra ovenstående afsnit hvor der blev set på hvilke fordele og ulemper de forskellige lommepengemodeller giver, er der blevet opstillet en række vurderingskriterier, som de forskellige modeller skal bedømmes ud fra. Derefter kan de bedste modeller findes, eller hvis muligt kan en eller flere modeller sammensættes og derved skabe en model, som vil opfylde kriterierne endnu bedre. Der gives vurdering på følgende:\\

\noindent\textbf{1) Økonomisk forståelse}
Som det blev forklaret i afsnit \ref{okonomipaaskemaet}, er det vigtigt for de unge at få en forståelse af, hvordan økonomi foregår, og hvordan de senere i deres liv vil blive nødt til at arbejde for at kunne tjene penge. 
Disse kriterier medtager også forståelsen for at der er forskel på hvor meget arbejde der lægges i de forskellige opgaver, og dermed også forskel på udbetalingerne.
Dette kriterie bedømmer hvor god mulighed de unge får for at lære om den økonomiske del ved hjælp af den valgte lommepengemodel.\\

\noindent\textbf{2) Husstandsbidrag}
Sammenhold i familien og husstanden. 
Det er vigtigt for de unge at lære at fungere i en familie og mindre grupper. Dette betyder derfor også at de unge skal hjælpe til i deres husstand.
Dette kriterie bedømmer i hvilken grad, den valgte lommepengemodel giver de unge muligheder for at udvikle sine evner til at hjælpe i familien og det at være del af en gruppe.\\

\noindent\textbf{3) Tidskrav}
Hvor meget tid lommepengemodellen kræver af de unge.
Dette kriterie kigger på hvor krævende det vil være for de unge, at benytte den valgte lommepengemodel, og dermed tage deres tid og koncentration fra andre emner, der kunne være vigtigere. Hvis de unge ikke får tid til at udvikle sig socialt og skolemæssigt, vil de få det langt sværere i deres senere liv, end hvis de har en dårlig forståelse for økonomi.
Derfor er det vigtigt at bedømme hvor balanceret de forskellige lommepengemodeller er, så det ikke tager alt de unges fritid.\\

\noindent\textbf{4) Enkelthed}
Dette kriterium vurderer hvor svært det vil være for børnene at benytte denne model, men også hvor svært det vil være for forældrene at opstille betingelserne og styre denne model.\\

\noindent\textbf{5) Digitaliserings relevans}
Dette kriterium vurderer i hvor høj grad det er relevant at fremstille en digitalisering af modellen. Dette vurderes ud fra hvor systematiseret modellen er. Ved systematisering forstås i hvor høj grad der er sammenhæng mellem det udbetalte beløb, og forventet udførelse. \\

\noindent Skemaet fungerer ved at der gives en bedømmelse fra 1-5 ved hvert kriterie, hvor 5 er bedst og 3 er neutral. Dermed vil hver lommepengemodel, have et antal point, som svarer til den samlede bedømmelse.\\


\begin{table}[htb]
   \centering
   \begin{tabular}{| l | l | l | l | l |} 
   \hline
   & \textbf{Model 1} & \textbf{Model 2} & \textbf{Model 3} & \textbf{Model 4} \\ \hline
   Økonomisk forståelse & 1 & 3 & 5 & 2 \\ \hline
   Husstandsbidrag & 1 & 3 & 3 & 4 \\ \hline
   Tidskrav (Højere er bedre) & 5 & 3 & 2 & 4 \\ \hline
   Enkelthed & 4 & 3 & 2 & 3 \\ \hline
   Digitaliserings relevans & 1 & 3 & 5 & 3 \\ \hline
   \textbf{I alt} & \textbf{12} & \textbf{15} & \textbf{17} & \textbf{16} \\ \hline
   \end{tabular}
   \caption{En vurdering af de fire lommepengemodeller.}
\end{table}

\noindent Herved ses, rapportens vurdering, af hvordan de forskellige modeller rangeres ud fra de opstillede kriterier.
Dog skal det pointeres, at denne vurdering prioriterer økonomiske læringsmæssige kvaliteter højere, end de sociale kvaliteter modellerne vil kunne give.
Dette betyder derved at modellerne 2, 3 og 4 får en bedre vurdering, i forhold til model 1, som ikke fokuserer på de økonomiske fordele. 
