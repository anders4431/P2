I dette afsnit bliver der konkluderet på hvilke elementer, fra netbankerne og de gennemgåede freelancesystemer, vi har valgt at implementere i vores lommepengesystem.

Fra bankerne er vi især blevet inspireret af deres måde at give kunden et overblik, dette har resulteret i at vi i vores program vil implementere noget lignende Spar Nord banks udseende. Dette betyder at der skal være en fane i programmet indeholdende en liste over indkomster og hævninger. Desuden vil der være to diagrammer i denne fane, det første er et cirkeldiagram som giver et overblik over hvor meget de forskellige pligter udgør af den samlede inkomst. Det andet er en graf der viser saldoen på kontoen i forhold til tid. Tidsintervallet der ska implementeres i vores program er også med inspiration fra netbankerne, vi har i første omgang valgt [[Afsluttets senere]]. Ydermere skal brugerne af programmet, ligesom bankkunder, have en vis procentdel i rente.

Fra freelancesystemerne er der ikke kun enkeltstående designmæssige ting vi har taget til os. Dette skyldes at hele grundidéen i det  valgte lommepengesystem, og disse freelancesystemer er den samme. Vi  vil, ligesom disse systemer, have en liste over ledige pligter. [[Pligterne]] på denne liste kan eventuelt have en kort beskrivelse skrevet af forældrene, derudover også hvor mange penge pligten giver. Valutaen i vores system er 1:1 i forhold til den danske krone, og kan ligesom visse freelancesystemer udbetales fra den virtuelle konto til brugerens rigtige bankkonto hvis brugeren anmoder om det. Yderligere er det blevet besluttet at barnet skal vælge opgaven i programmet og så derefter løse den, det er altså ikke meningen at barnet kan løse en opgave og så først derefter skrive dette ind i systemet. Dette er valgt på baggrund af at søskende ikke skal kunne tage æren for en opgave et andet familiemedlem har løst.

I øvrigt er vi opmærksomme på at lave layoutet så simpelt og brugervenligt som muligt, da det er børn og ikke nødvendigvis tekniske forældre der er programmets målgruppe. 

Vi har fravalgt at lave søjlediagrammer, dette vil muligvis høre til det videre arbejde. Yderligere har vi fravalgt at børnene kan lave transaktioner mellem hinanden, dette sker på baggrund af at ældre søskende ikke skal have mulighed for at udnytte deres yngre søskende.




