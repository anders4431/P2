Ud fra den forudgående problemanalyse, er der blevet argumenteret for, at børn og unge har en utilstrækkelig forståelse for økonomiske elementer som renter, lån og opsparinger, samt en manglende forståelse for økonomisk ansvar. Alt dette er tilfældet, selvom størstedelen af børn og unge modtager lommepenge fra deres forældre, samt undervisning i økonomiske elementer i deres skole, som begge gerne skulle have hjulpet dem til at have viden om økonomi. Derudover er der også blevet undersøgt og vurderet de forskellige versioner af lommepengemodeller, som forældre bruger til at udlevere og administrere lommepenge. På baggrund af dette problem, og den forudgående analyse af problemet, er der blevet udarbejdet følgende problemformulering:

\begin{itemize}

	\item Hvordan kan man lave et system a la et netbanksystem, på en datalogisk måde, som er baseret på lommepenge og pligter?
	\item Hvordan kan man lære børn og unge omkring økonomi, samt give en bedre forståelse for økonomi, ved hjælp af en datalogisk løsning?

\end{itemize}

\subsection{Systemdefinition}
Et IT system, der bruges i en husholdning til at oprette, administrere og behandle arbejdet omkring pligter, udført af børnene i husholdningen, med hovedvægt på at lære børnene basal økonomisk forståelse. Systemet skal primært være et organisatorisk samt lærende system for økonomi, heriblandt mulighed for statistik som grafer, men sekundært også gøre brug af “Freelance” ansættelsesformen, for øget frihed til børnene omkring hvordan de håndterer deres arbejde, i form af pligter, og økonomi. Systemet skal være brugbart på en PC-platform egnet for husholdninger, samt husholdninger uden større teknologisk forståelse. Systemet  skal gøre forskellige værktøjer tilgængelige, så systemet kan blive sat op alt efter forældrenes initiativ og hvad de ønsker ud af systemet. Derudover skal der være en form for sikkerhed og fortrolighed omkring de penge børnene tjener og har.

Ud fra dette vil der blive arbejdet hen mod en prototype, baseret på objektorienteret programmering.