Til dette projekt skal der udarbejdes en objektorienteret løsning i form at et program, som kan administrere og behandle lommepenge i en husholdning. Systemet skulle gerne gøre det muligt for hvem som helst at bruge systemet, i form af interaktion med brugeren, guidende tooltips, eller andet. Systemet skulle også kunne virke som et lærende værktøj, og lære børn og unge omkring økonomiske elementer. Der skal være mulighed for mere detaljeret oversigter om ens data, i form af grafer eller lignende, men det skal ikke være en nødvendighed for at gøre brug af systemet. Derudover skal der også blive lavet en passende brugergrænseflade, som kan understøtte læringen af økonomi, samt være guidende for brugeren.

\begin{itemize}
	\item Funktionelle Krav
	\begin{itemize}
		\item Systemet skal tage højde for transaktioner af virkelige penge.
		\item Det skal være muligt at generere grafer for data i systemet, som f. eks. indtægter og antal pligter lavet af samme slags.
		\item En god måde at gemme og administrere transaktioner og data, som f. eks. ved hjælp af en database.
		\item Prototypen skal være brugbar og skal kunne køre på en Windows computer.
		\item Systemet skal have en passende brugergrænseflade.
	\end{itemize}
	\item Ikke-Funktionelle Krav
	\begin{itemize}
		\item Produktet skal afleveres sammen med en tilhørende rapport, og har en fælles deadline den 22. maj 2013.
		\item Programmet skal skrives i programmeringssproget C#.
		\item Systemet skal udvikles ud fra objektorienteret programmering.
		\item Systemet og projektet skal understøtte de krav og læringsmål som studieordningen foreskriver.
	\end{itemize}
	\item Løsningsmål
	\begin{itemize}
		\item Der skal være tilstrækkelig brugerinteraktion og brugervejledning, til at hjælpe brugeren med at bruge systemet. F.eks. med tooltips.
		\item Interaktionen og brug af systemet skal kunne give en forståelse for økonomiske elementer.
		\item Det skal ikke være nødvendigt, som bruger, at skulle til at gå helt ned i detaljen med grafer osv., for at gøre brug af systemet. Muligheden skal dog være der.
		\item System skal have sigende en brugergrænseflade, for at understøtte simpliciteten, og bruger interaktionen.
	\end{itemize}
\end{itemize}