%\subsection{Hvor meget kan hver aldersgruppe lære?}
I dette afsnit vil der blive undersøgt hvornår børn lærer de færdigheder, som skal til for at kunne styre en privatøkonomi.

Undervisningsministeriet udgav i 2009 en serie hæfter med titlen Fælles Mål 2009, som beskriver hvad der forventes at eleverne lærer i de enkelte fag. En del af de Fælles Mål er, at beskrive trin for trin hvad der forventes af eleverne i de forskellige forløb i folkeskolen. Folkeskolen er inddelt i 4 forløb;
\begin{itemize}
\item 1.-3. klasse er forløb 1.
\item 4.-6. klasse er forløb 2.
\item 7.-9. klasse er forløb 3.
\item 10. klasse er forløb 4.
\end{itemize}

\emph{Efter 6. klasse}
%linebreak here
I faget matematik er det forventet at elever efter 6. klasse skal kunne regne med procenter, 
nærmere bestemt skal de: “kende procentbegrebet og bruge enkel procentregning”\cite{FallesMalMatematik}
Dette er naturligvis vigtigt da det er en forudsætning for at kunne regne med og forstå renter.

%en blank linje her...?
\emph{Efter 9. klasse}
%linebreak here
I matematik er det forventet at elever efter 9. klasse skal kunne regne med lønopgørelser, skatteberegninger og rentebegrebet.\cite{FallesMalMatematik} Sammen med rentebegrebet hører blandt andet også opsparing, låntagning og kreditkøb.
I samfundsfag, som i folkeskolen starter i 7- klasse, skal elever efter endt forløb kunne blandt andet: “Redegøre for det økonomiske kredsløb og markedsmekanismer”\cite{FallesMalSamfundsfag}
Dette indebærer blandt andet at eleverne skal kunne forstå begreber som lån, lønseddel, opsparing og husholdning.

[[En eller anden form for konklusion??;
Vi kan se at unge skal, i skolen, lære om procenter omkring år 11-12, og derfor... bla bla ]]
