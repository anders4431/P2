Der findes mange forskellige typer børn og unge, 
og derfor er det også meget forskelligt hvordan 
disse unge lærer bedst. Dette betyder også at 
der findes en række forskellige modeller for 
hvordan lommepenge kan gives. I nogle modeller 
skal børnene selv bidrage til de huslige pligter 
for derved at gøre sig fortjent til 
lommepengene, hvorimod andre modeller benytter 
lommepenge som et tilskud, børnene og de unge 
får et antal gange om året eller måneden. Hver 
enkelt model har dog sine egne fordele og 
ulemper, som vil blive fremhævet i dette afsnit. 
De forskellige modeller er fundet hos danske 
bank.

\section{Model 1: Det rene tilskud}
Ved denne model, vil børn og unge få det samme 
beløb, et antal gange om måneden eller året. 
Dette kan f.eks. være starten til hver måned 
eller hver uge. Denne form for model kræver ikke 
at børnene eller de unge har pligter.

En af fordelene ved denne model, er at barnet 
eller den unge får mulighed, for at koncentrere 
sig om vigtigere ting på daværende tidspunkt, 
som ikke vil være muligt at lære eller gøre 
senere. Dette kan for eksempel være det sociale 
liv eller lektier og skole, som nogen vil finde 
vigtigere på et sådant tidspunkt.

En ulempe ved denne model, er at manglen på 
pligter kan give barnet det indtryk, at penge er 
en selvfølge, som ikke behøves at skaffes gennem 
arbejde. Dette kan gøre at barnet senere hen, 
vil have svært ved at acceptere at skulle 
arbejde for at få penge, og derved skabe et 
såkaldt “luksusbarn” som forventer at blive 
betalt uden at give noget igen, i form af 
arbejde eller studieengagement.

\section{Model 2: Lommepenge inkluderer 
pligter}
Denne model bygger oven på den forrige model ved 
at inkludere faste pligter til børnene eller de 
unge, som skal klares, for at lommepengene 
bliver udbetalt.

En fordel ved denne model, er i kontrast med 
modsatte model, en forståelse for at man skal 
arbejde for at få penge, og derved skaber et 
bedre forhold til brugen af pengene i sig selv. 
Et sådant forhold vil kunne hjælpe den unge 
person, når han eller hun skal stå på egne ben 
og styre sin egen økonomi, når denne person 
flytter hjemmefra.

En anden fordel ved denne model, er at det gør 
børnene og de unge opmærksom på kravene for at 
en husstand kan fungere og derved giver en 
helhed i familien, som børnene bliver en del af. 
Et sådant sammenhold vil ikke kun styrke 
familien, men også give børnene en bedre fordel 
ved fremtidige sociale forhold, hvor der også 
stilles krav til dem.

En ulempe ved denne model, er at børnene kan 
vokse op med en forventning om udbetaling, for 
hver enkelt gang de yder hjælp i 
familiesammenhænge. Dette kan bevirke at børnene 
enten kan begynde at kræve penge for alt hvad de 
laver, eller at disse børn simpelthen ikke vil 
lave noget udover, de pligter de blev påsat i 
starten. Et sådant forhold til pligter og krav, 
kan give en destruktiv adfærd, som ikke vil 
gavne når barnet skal videre i sin dannelse.

\section{Model 3: Akkordmodellen}
Akkordmodellen bygger på forholdet mellem penge 
og arbejde, som der bliver benyttet i det 
virkelige liv. Penge vil blive udbetalt, for 
hver pligt i husstanden, der bliver løst. 
Beløbet som der udbetales kan være et fast beløb 
for alle pligter, eller det kan være et 
varierende beløb, som ændres efter varigheden af 
og sværhedsgraden pligten.

En af fordelene ved akkordmodellen, er dens 
mulighed for at give børnene og de unge, et 
indblik i den virkelige verdens økonomi, hvor 
der er sammenhæng mellem arbejdet og løn. Dette 
betyder således at de unge i fremtiden vil have 
nemmere ved at kunne se pointen i jobs, og 
derved skabe en tidlig lyst til at blive 
selvforsynende og dermed lære endnu mere om 
deres fremtidige økonomi.

En ulempe ved denne model, er at dennes krav til 
udbetaling for hver enkelt pligt, kan give 
barnet mistillid til familiens forhold, og at 
det vil kræve penge for at være en del af denne. 
Dette kan være en dårlig indstilling for 
fremtidige sociale udfordringer.

\section{Model 4: Lommepenge et, pligter 
noget andet}
Denne model ser på lommepenge og huslige pligter 
som to forskellige ting, som ikke skal have 
nogen sammenhæng. Dette medfører dermed at 
børnene, vil få lommepenge med et fast interval, 
eller børnene har brug for det. Anden del af 
denne model, betyder så også at det kræves af 
børnene at et antal pligter bliver udført, men 
mangel på denne udførsel, vil ikke påvirke 
lommepengene.

En fordel ved denne model, er at det giver 
barnet et sammenhold med familien, og en 
forståelse af hvordan en husstand har brug for 
alles hjælp, for at kunne fungere, uden at der 
skal gives lommepenge for de ting hver person 
hjælper med. Dette vil også kunne hjælpe 
fremover, ved sociale tidspunkter, hvor penge 
ikke er et punkt, mens sammenholdet i en mulig 
gruppe er.

En ulempe som denne model giver, er det svage 
billede af økonomien, barnet vil vokse op med. 
Barnet vil se penge som en selvfølge og ikke se 
sammenhængen mellem dets pligter og 
lommepengene. Dette kan være med til at skabe et 
dårligt forhold til penge i sig selv, og derved 
skabe problemer for barnet eller den unge i sit 
senere liv.

Det ses at der ved hver model, både er fordele, 
men også ulemper, der skal tages højde for, når 
forældre vælger hvilken form for lommepenge, 
deres børn skal have.


