Noget, der kunne være værd at kigge på i projektet, er, hvordan man kan lære børn og unge omkring økonomi på en passiv måde, sådan, at deres udvikling ikke bliver forstyrret. Dette skal forstås som, at børn skal have love til at være børn. En dreng på 9 år burde ikke behøve at bruge endeløse timer, for at få sin økonomi til at hænge sammen. Disse økonomiske erfaringer skulle gerne blive erfaret passivt, mens barnet får lov til at nyde sin barndom. Der er dog nogle problemer med at undersøge, og tage højde for, dette.

Det er svært at vurdere om et barn har fået noget som helst ud af denne passive læring, da det vil tage flere år før man kan se om det har haft en effekt, når barnet er ældre og konsekvenserne er større. Det er ikke muligt at sige med sikkerhed, om et barn har fået noget ud af den passive læring, medmindre man har adskillige år til at undersøge om det havde en effekt. Rækkevidden og tiden for projektet er ikke stort nok, til at om den passive læring vil give den ønskede effekt på barnet, når det engang er blevet ældre. På grund af dette vil der blive set bort fra dette i projektet.