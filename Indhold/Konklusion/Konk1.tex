I dette afsnit vil projektets konklusion fremgå. Det er naturligvis problemformuleringen der skal besvares, det er derfor denne der bliver kigget tilbage på.

Den første del af problemformuleringen lyder;\\
\textit{Hvordan kan man lave et system a la et netbanksystem, på en datalogisk måde, som er baseret på lommepenge og pligter?}

Hertil er der blevet udviklet et elektronisk lommepenge-system, som kan bruges af familier til at administere lommepenge til børnene. Under problemanalysen, blev det klart at der i danske familier bliver brugt flere forksellige lommepengemodeller. Dette var selvfølgelige noget der skulle tages højde for under problemløsningen. Det endte dog med at den model, som programmet implementere på nuværende tidpsunkt er akkordmodellen. Det vil sige at forældrene kan udgive nogle pligter, som de ønsker skal udføres og give en pris på hvor mange penge man kan tjene på pligten. Når et barn så har lavet en pligt, kan de markere at den er løst. Herefter kan forældrene kontrollere om pligten er udført og derefter indsætte den virtuelle løn på barnets konto. Vi kan konkludere at der er blevet udviklet et system, som er baseret på lommepenge og pligter.

Den næste del af problemformuleringen lyder;\\
\textit{Hvordan kan man lære børn og unge omkring økonomi, samt give en bedre forståelse for økonomi, ved hjælp af en datalogisk løsning?}

Til at besvarer dette spørgsmål, er der gennem problemanalysen blevet undersøgt hvilke grunde der er til at unge mennesker, har svært ved at styre deres økonomi. Her blev der på baggrund af forskellige kilder antaget at børn lærer for lidt om privatøkonomi gennem deres opvækst. Eksempelvis ved  37% danskere mellem 18-27 år ikke hvad ordet "rente" betyder, ligesom at godt 50.000 unge er registeret i RKI , som dårlige betalere. 
I programmet indår elementer som læring om opsparing og læring om hvor penge kommer fra. På sigt skal elementer som rente også implemteres. Da vi i projektforløbet ikke har haft tid til at teste programmet på børn, kan vi ikke konkludere om løsningen rent faktisk kan give unge en bedre forståelse for økonomi. Til en sådan test skulle man bruge en tidsperiode på mellem 10 og 15 år. For hvis man tester programmet på børn i 5-10 års alderen, kan man først se resultatet når de er i 20-25 års alderen. 

En samlet konklusion må være at der er blevet udviklet et stykke software, som er baseret på lommepenge og pligter, men at der ikke er baggrund for at konkludere hvorvidt børn kan lærer noget økonomisk af systemet.