Der bliver kigget tilbage på problemformuleringen og det konkluderes hvorvidt denne er opfyldt.  

Den første del af problemformuleringen lyder;\\
\textit{Hvordan kan man lave et system a la et netbanksystem, på en datalogisk måde, som er baseret på lommepenge og pligter?}

Hertil er der blevet udviklet et elektronisk lommepenge-system, som kan bruges af familier til at administrere lommepenge til børnene. Under problemanalysen blev det klart at der i danske familier bliver brugt flere forskellige lommepenge-modeller. Dette var selvfølgelig noget der skulle tages højde for under problemløsningen. Det endte dog med, at den eneste model, som programmet implementerer på nuværende tidspunkt er akkord modellen. Det vil sige, at forældrene kan oprette pligter, som de ønsker udført og sætte en pris på, hvor mange penge man kan tjene på at løse pligten. Når et barn så har lavet en pligt, kan de anmode om, at den er løst. Herefter kan forældrene kontrollere, om pligten er udført og derefter indsætte lønnen på barnets virtuelle konto. Vi kan konkludere, at der er blevet udviklet et system, som er baseret på lommepenge og pligter.

Den næste del af problemformuleringen lyder;\\
\textit{Hvordan kan man lære børn og unge omkring økonomi, samt give en bedre forståelse for økonomi, ved hjælp af en datalogisk løsning?}

Til at besvare dette spørgsmål er der gennem problemanalysen blevet undersøgt hvilke grunde der er til, at unge mennesker har svært ved at styre deres økonomi. Her blev der på baggrund af forskellige kilder antaget, at børn lærer for lidt om privatøkonomi gennem deres opvækst. 
I programmet indgår elementer som læring om opsparing og læring om hvor penge kommer fra. På sigt skal elementer som rente og eventuelt skat også implementeres. Da vi i projektforløbet ikke har haft tid til at teste programmet på børn, kan vi ikke konkludere om løsningen rent faktisk kan give unge en bedre forståelse for økonomi. Til en sådan test skulle man bruge en tidsperiode på mellem 10 og 15 år. For hvis man tester programmet på børn i 5-10 års alderen, kan man først se resultatet når de er i 20-25 års alderen.

En samlet konklusion må være, at der er blevet udviklet et stykke software, som er baseret på lommepenge og pligter, men at der ikke er baggrund for, at konkludere hvorvidt børn kan lærer noget økonomisk af systemet.
