I dette afsnit bliver de ting der ikke blev tid til i denne projektperiode gennemgået. Der har løbende været mange idéer oppe at vende, nogle blev sorteret fra fordi de ikke var gode nok, og andre fordi vi ikke havde tid nok. 

I brainstormingsfasen havde vi flere idéer, som sidenhen er blevet forkastet. Som udgangspunkt skulle programmet være en konsol applikation, men dette ville ikke give meget mening i forhold til vores projekt. Yderligere var det på tale, at gøre programmet beregnet til mere end bare den enkelte familie. Man kunne have lavet programmet således at børnene i en klasse, eller en anden form for forsamling, kunne have et fælles program. Dette ville medføre, at brugerne skulle være i stand til at handle med hinanden via deres virtuelle konto, og påtage sig pligter for andre end deres egne forældre. Dette ville muligvis øge programmets popularitet, og dermed få flere brugere. Dette forslag blev dog ret hurtigt forkastet, da vi var bange for at de smarteste elever ville udnytte dette system til at franarre kammeraterne deres lommepenge. Desuden fandt vi det som en dårlig idé, at forældre skulle betale deres børn for potentielt at lave pligter i andres husstande.

Hvis der havde været mere tid, ville der have været lagt flere kræfter i 'statistik'-fanen i programmet. Det kunne være en fordel hvis brugeren havde mulighed for at vælge flere forskellige diagrammer, da hvert diagram har hver sin fordel. Det var tiltænkt at man i denne fane kunne vælge to standard diagrammer, således at man fik sine favorit diagrammer frem på siden, i stedet for at have to diagrammer påtvunget, ligesom det er nu. Udover tidspres, så har vi også valgt dette fra, for at holde programmet så brugervenligt og simpelt som muligt. Vi har ikke kunnet vurdere hvorvidt denne feature ville skabe mere forvirring end gavn hos brugeren. Desuden ville det også være ideelt, hvis brugeren havde en såkaldt \textit{'slider'} eller en anden mulighed for at definere tidsintervallet.

Generelt set har mange af de fravalgte ting, der er på tegnebrættet noget med kompleksitet og brugervenlighed at gøre. Netop brugervenlighed var kernen i næste ting, der var på listen over ting, der skal laves inden programmet publiceres. Det er meningen, at der i det endelige program skal være en tutorial, som gennemgår de forskellige knapper og faner, så barnet har bedst mulighed for at lære mest muligt af programmet.

For at give programmet en forlænget levetid, forstået på den måde, at børnene også får noget ud af programmet når de bliver lidt ældre, har vi tænkt os at implementere flere økonomiske aspekter. Dette kunne for eksempel være at indføre en slags \textit{skat} og \textit{gebyrer}.

I det færdige program er det tiltænkt, at forældrene skal have mulighed for at vælge mellem nogle forskellige af de gennemgåede lommepengemodeller. Til at starte med, er der kun blevet implementeret én model, akkordmodellen, men da forældre har forskellige meninger om hvordan børnene skal opdrages mht. økonomi, er vi nødt til at give dem muligheden for at ændre modellen, således at vi ikke på forhånd har sorteret en række potentielle brugere fra.

Ydermere bør forældrene have mulighed for at tildele børn pligter, så ‘dovne’ børn ikke bare kan sige at de ikke gider lave noget. Deslige kunne man forestille sig at familierne kunne være interesserede i at uddele faste pligter til børnene, så man ikke hver dag manuelt er nødt til at oprette en ‘opvask’ for eksempel. Dette betyder at programmet skal indeholde en slags kalender, som holder styr på de faste pligter børnene har. Et eksempel herpå kunne netop være at barnet \textit{skal} tage opvasken hver onsdag. I forlængelse af dette kunne man forestille sig, at børn med faste pligter kunne være interesserede i at bytte pligter internt med hinanden.