I dette afsnit bliver de ting der ikke blev tid til i denne projektperiode gennemgået. Der har løbende været mange idéer oppe at vende, nogen blev sorteret fra fordi de ikke var gode nok, og andre fordi vi ikke havde tid nok til disse. Det er sidstnævnte, som vi gennemgår her.

Hvis der havde været mere tid ville der have været lagt flere kræfter i statistikfanen i programmet. Det kunne være en fordel hvis brugeren havde mulighed for at vælge flere forskellige diagrammer, da hvert diagram har hver sin fordel. Det var tiltænkt at man i denne fane kunne vælge to standard diagrammer, således at man fik sine favorit diagrammer frem på siden, i stedet for at have to diagrammer påtvunget, ligesom det er nu. Udover tidspres, så har vi også valgt dette fra, for at holde programmet så brugervenligt og simpelt som muligt. Vi har ikke kunnet vurdere hvorvidt denne feature ville skabe mere forvirring end gavn hos brugeren.[[Yderligere er det tiltænkt at linjediagrammet skal vise en graf over udviklingen af den samlede saldo, lige nu viser linjediagrammet hvad der bliver tjent dag for dag.]] Desuden ville det også være ideelt hvis brugeren havde en såkaldt textit{"slider"} eller en anden mulighed for at definere tidsintervallet

Generelt set har mange af de ting der er på tegnebrættet noget med kompleksitet og brugervenlighede at gøre. 
Netop brugervenlighed var kernen i næste ting der var på listen over ting der skal laves end programmet publiceres. Det er meningen at der i det endelige program skal være en turtorial, som gennemgår de forskellige knapper og faner, så barnet har bedst mulighed for at lære mest muligt af programmet. 

For at give programmet en forlænget levetid, forstået på den måde, at børnene også får noget ud af programmet når de bliver lidt ældre, har vi tænkt os at implementere flere økonomiske aspekter. Dette kunne for eksempel være at indføre en slags textit{skat} og textit{gebyrer}. 

I det færdige program er det tiltænkt, at forældrene skal have mulighed for at vælge mellem nogle forskellige af de gennemgåede lommepengemodeller. Til at starte med er der kun blevet implementeret én model, akkordmodellen, men da forældre har forskellige meninger om hvordan børnene skal opdrages mht. økonomi, er vi nødt til at give dem muligheden for at ændre modellen, således at vi ikke på forhånd har sorteret en række potientielle brugere fra.




