Vores løsning har nogle mangler, der bevirker at programmet ikke vil virke hensigtsmæssigt i en familie. Dette skyldes blandt andet projektets begrænsede tidsperiode. Inden programmet ville blive udgivet, er der nogle ting der skulle udvikles:

Programmet virker, som det er nu kun på en enkelt computer. Det vil sige at brugerne og admins, børnene og forældrene, er nødt til at bruge samme computer, hvilket ikke gør programmet ret fleksibelt. Dette underbygges af at flere og flere børn får deres egen computer, mobiltelefoner og tablets, dette endda i en tidligere alder end før.\cite{MobilAlder}

En anden ting der ville gøre programmet mere fleksibelt ligger i forlængelse af førnævnte punkt. Det ville være idéelt, hvis der udover programmet blev udviklet en app til smartphones og tablets, således at børnene kan påtage sig en pligt, uden at være afhængige af en tændt, måske stationær, computer. Ligeledes kunne man integrere en web-baseret løsning i systemet, således at så længe man har adgang til internettet, så vil man kunne gøre brug af systemet. Disse tiltag ville gøre systemet langt mere fleksibelt.

Man kunne forestille sig, at dette program ikke vil fungere optimalt i en familie med kun ét barn. I denne forbindelse kunne man lave en udvidelse af programmet, således at man kunne inddrage flere personer end bare forældrene som 'arbejdsgivere'. Det kunne være bedsteforældre eller andre der bor i nabolaget. Man kunne forestille sig at disse kunne have interesse i at få slået deres græs eller lignende.

