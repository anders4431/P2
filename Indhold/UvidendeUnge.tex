
Ifølge en undersøgelse, lavet af YouGov Zapera \cite{DanskeB1}, har unge danskere store økonomiske problemer. 37\% af de adspurgte danskere mellem 18- og 27 år tilkendegav i undersøgelsen, at de ikke ved hvad ordet ”rente” betyder, over halvdelen kan ikke udpege den billigste blandt tre simple lånemodeller og rigtig mange af de unge har ofte overtræk på deres konti. Dog siger Danske Bank at de ældre generationer ikke er meget bedre, hvilket kunne betyde at de unge arver deres forældres dårlige vaner.  \\
\\
Samtidigt drukner de unge i stor gæld. En ny opgørelse fra Statens Administration viser, at unge skylder staten over 22 milliarder i SU-gæld. Denne gæld kommer af, at knap 55.000\cite{dr.dk} nuværende og tidligere studerende har studie-lån, som ikke er afbetalt på. Ifølge økonomistyrelsen havner unge i disse problemer, fordi de efter endt studie, hvor SU-lånet skal tilbagebetales, har svært ved at komme i job\cite{jobindex}. Faktisk har mange så svært ved at betale gælden tilbage, at sagerne ryger videre til inddrivelse hos skat\cite{BusinessDK1}.\\
\\
Herudover er godt 50.000 unge registreret i RKI som dårlige betalere, hvilket er flere end nogensinde ifølge Kim Bach fra Experian. Danske Studerendes Fællesråd og Danske Bank har undersøgt, hvorfor så mange unge har gæld, som de ikke kan betale tilbage. Ifølge dem er årsagen, at unge ved for lidt om privatøkonomi. Danske Studerendes Fællesråd foreslår, at unge bør få gratis rådgivning, når de starter på et SU-lån. \cite{dr.dk}.
