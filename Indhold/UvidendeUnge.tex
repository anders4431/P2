 Ifølge en undersøgelse, lavet af YouGov Zapera \cite{DanskeB1}, har unge danskere store økonomiske problemer. 37\% af de adspurgte danskere mellem 18- og 27 år tilkendegav i undersøgelsen, at de ikke ved hvad ordet ”rente” betyder, over halvdelen kan ikke udpege den billigste blandt tre simple lånemodeller, og rigtig mange af de unge har ofte overtræk på deres konti. Dog siger Danske Bank, at de ældre generationer ikke er meget bedre, hvilket kunne betyde at de unge arver deres forældres dårlige vaner. Det kunne på den anden side også være et tegn på, at vi i vores opvækst har brug for at lære mere om økonomi. Hvis man, som forældre, ikke ved så meget om det, har man meget svært ved at lære det videre til børnene. Dette giver altså anledning til, at man allerede gennem sin opvækst, har brug for at forstå hvad rente er, og hvad der sker når man bruger penge man ikke har, altså låner.    \\
\\
Samtidigt drukner de unge i stor gæld. En ny opgørelse fra Statens Administration viser, at unge skylder staten over 22 milliarder i SU-gæld. Denne gæld kommer blandt andet af, at knap 55.000 \cite{dr.dk} nuværende og tidligere studerende har studie-lån, som ikke er afbetalt på. Ifølge økonomistyrelsen havner unge i disse problemer, fordi de efter endt studie, hvor SU-lånet skal tilbagebetales, har svært ved at komme i job\cite{jobindex}. Faktisk har mange så svært ved at betale gælden tilbage, at sagerne ryger videre til inddrivelse hos skat\cite{BusinessDK1}. I virkeligheden er der altså et ret stort problem med SU-lånene. For mange unge, som har svært ved at få de økonomiske ender til at nå sammen under studietiden, er SU-lånene meget attraktive. De har en meget lav rente, og så skal de først betales tilbage ved starten af det andet kalenderår, efter endt studie. Alle studerende regner selvfølgelig med at have job til den tid, hvorfra tilbagebetaling ikke burde være et problem. Hvis dette modsat ikke sker, havner de fleste på kontanthjælp, eller på dagpenge. Og med en sådan indkomst, kan det være utrolig svært at betale en gæld tilbage. Derfor kan det være farligere at optage et lån end den gennemsnitlige studerende måske regner med. Der er omvendt også den mulighed, at studerende efter endt studie, optager huslån, billån, eller lignende og vælger at afbetale på disse istedet for deres SU-lån, da SU-lån er langt billigere at have end banklån. Det er dog stadig relevant at man i en tidlig alder får kendskab til lån og hvordan disse fungerer.  \\
\\
Herudover er godt 50.000 unge registreret i RKI som dårlige betalere, hvilket er flere end nogensinde, ifølge Kim Bach fra Experian. Danske Studerendes Fællesråd og Danske Bank har undersøgt, hvorfor så mange unge har gæld, som de ikke kan betale tilbage. Ifølge dem er årsagen, at unge ved for lidt om privatøkonomi. Vi ser dog også den mulighed, at unge simpelthen ikke får et job, og derfor ikke har råd til at afbetale. Danske Studerendes Fællesråd foreslår, at unge bør få gratis rådgivning, når de starter på et SU-lån. \cite{dr.dk}. En anden mulighed, er at man giver unge en forståelse for lån og gæld, gennem deres opvækst, og lader dem forstå, at det har en konsekvens, at man ikke får betalt de penge man skylder, til tiden. Samtidigt bør man sørge for, at alle kender fordelen ved at spare op. Således kan man opmuntre unge mennesker til at tage et sabbatår inden studiestart, hvor de både kan "fjumre" og tjene penge til en god opsparing. På den måde kunne mange blive fri for overhovedet at tage et lån.
Det er dog værd at understrege, at politikerne går ind for at man kommer hurtigst muligt igennem systemet, så man kan komme ud på den anden side og tjene penge til samfundet. Derfor er de ikke vilde med sabbatår, og har således lavet forskellige initiativer herimod. Blandt andet den aktuelle og meget omtalte SU-reform \cite{SUreform}, og at man kan gange sit gennemsnit op, hvis man starter på en videregående uddannelse inden for 2 år \cite{SabbatAar}.
 