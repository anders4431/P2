Uvidenhed angående økonomi for unge danskere

Ifølge en undersøgelse lavet af YouGov Zapera \cite{DanskeB1} har unge danskere store økonomiske problemer. 37\% af de adspurgte danskere mellem 18 og 27 tilkendegav i undersøgelsen at de ikke ved hvad ordet ”rente” betyder, ligesom at over halvdelen ikke kan udpege den billigste blandt tre simple lånemodeller. Ligeledes afslørede undersøgelsen at rigtig mange af de unge ofte har overtræk på deres konti. Dog siger Danske Bank at de ældre generationer ikke er meget bedre, hvilket kunne betyde at de unge arver deres forældres dårlige vaner.  

Samtidigt drukner de unge i stor gæld. En ny opgørelse fra Statens Administration, det tidligere Økonomistyrelse, viser at unge skylder staten over 22 milliarder i SU-gæld. Denne gæld har de unge så svært ved at betale tilbage, at sagerne ryger videre til inddrivelse hos skat\cite{BusinessDK1}. Hertil kommer at 7\% af unge danskere mellem 18 og 27 er registreret i RKI som dårlige betalere\cite{BusinessDK2}.
