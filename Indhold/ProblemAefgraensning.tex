Dette afsnit vil prøve at afgrænse projektets problemfelt, baseret på den foregående problemanalyse, for at finde en mere tydelige, specificeret og relevant målgruppe, som den efterfølgende del af projektet vil prøve at behandle. [[Afgrænsede elementer kan blive inkluderet senere i projektet, hvis det viser sig, at tiden tillader det.]]

Problemanalysen har vist, at børn og unge har en utilstrækkelig forståelse for økonomi, selvom størstedelen modtager både undervisning og lommepenge, som gerne skulle have hjulpet dem til at få en bedre forståelse. Ud fra afsnit \ref{ØkonomiSke} og \ref{LommeStat}  kan man umiddelbart se, at børn modtager undervisning i emner relevant til økonomi omkring slutningen af andet forløb i folkeskolen, som procentberegning. Derudover kan man se ud fra statistikkerne, at indføring af lommepenge først rigtig slår ud omkring 9 til 11 års alderen, hvor antallet sprang fra 38 procent til 78 procent. Ud fra dette kan man se, at børn først rigtig begynder at lære omkring økonomi når de er lidt under 10, hvor de også er på vippen til at lære om økonomisk relevante emner i skolen. Et supplement til læring af økonomi vil derfor være mest nyttigt omkring den samme tid, og derfor vil der blive afgrænset til at have et fokus startende på denne aldersgruppe.

I afsnit \ref{LommeModeller} blev der undersøgt, hvilke forskellige lommepengemodeller der findes, fordele og ulemper ved disse modeller, samt om hvordan de bliver brugt til at administrere lommepenge i virkelige husholdninger. Derudover bliver der i det senere afsnit \ref{ModelVurdering} yderligere evalueret på disse lommepengemodeller udfra bestemte kriterier. Denne vurdering viser, at især ‘Akkordmodellen’ og ‘Lommepenge et, pligter noget andet’ står ud som de bedste af modellerne. Ud fra dette vil der blive afgrænset til at der skal implementere en eller muligvis flere modeller, baseret på de to førnævnte modeller, eller kombinationer mellem disse og eventuelle andre modeller. Helt præcis hvilke modeller der bliver taget i brug vil blive dækket yderligere i det kommende Design afsnit[[Ref når kapitel findes]], som skal prøve at finde frem til den bedste model, eller modeller, at implementere.

Til sidst er der afsnittet \ref{Valuta}, som gør rede for fordelene og ulemperne, ved at basere en løsning på rigtige lommepenge, eller et fiktivt valutasystem. Begge sider af sagen har både mange positive og negativesider, så det er ikke helt klart hvilken af dem der vil fungere bedst i praksis, og give det bedst mulige svar på problemstillingen. Optimalt ville man prøve at implementerer begge muligheder, for at give flere muligheder til hvad brugeren kan med systemet. Dilemmaet her er at at de to former er vidt forskellige, og skal derfor også implementeres og udvikles med vidt forskellige forhold i tankerne, som der ikke nødvendigvis er tid til. Derfor vil der umiddelbart blive afgrænset til et system baseret på rigtige lommepenge, som virker til at være bedre gearet til at lære om økonomi.