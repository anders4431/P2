Økonomi betyder ifølge politikens retskrivnings og betydningsordbog \textit{"den måde den enkelte person producerer og forbruger sine værdier på"}. Nyere undersøgelser viser, at især unge mennesker har svært ved at holde styr på deres pengeforhold. For eksempel drukner mange unge i dag i gæld, og mange bliver registreret i RKI som dårlige betalere.
Lommepenge er ofte første gang børn får kendskab til det, at råde over egne penge. Måske kunne man udnytte lommepengene til at give folk en forståelse for økonomi allerede gennem deres opvækst, og dermed opnå at de er mere økonomisk ansvarlige, når de bliver ældre. Det er netop denne problematik der danner bund for dette projekt.\\
\\
Til start analyseres problemområdet, hvor der blandt andet kigges på en undersøgelse af unges økonomiske viden og der undersøges hvor meget børn lærer om økonomi i folkeskolen. Derudover bliver der kigget på, hvordan lommepenge og pligter hænger sammen i danske hjem og herefter ses der på eksisterende løsninger, hvorefter der fremgår en problemformulering for projektet.

Derved bliver det fastlagt hvilket problem, der skal løses, og hvad der er vigtigt at fokusere på i løsningen. Efterfølgende kommer et kapitel omhandlende designet af programmet, som handler om hvilke designmønstre, der bruges til udvikling af programmet og hvordan opbygningen skal se ud. Derudover beskrives brugergrænsefladen til programmet. Efterfølgende findes et kapite,l der udelukkende handler om programmet, herunder hvordan databaser fungerer, hvordan login virker og hvordan grafer indgår i programmet. Slutteligt konkluderes der på problemformuleringen og der beskrives hvad der kunne arbejdes videre med, hvis der havde været mere tid til projektet. 

