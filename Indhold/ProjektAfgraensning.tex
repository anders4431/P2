Dette afsnit vil afgrænse projektets problemfelt, baseret på den foregående problemanalyse, for at finde en mere tydelige, specificeret og relevant målgruppe.

Der blev argumenteret for, at børn og unge har en utilstrækkelig forståelse for økonomi, selvom størstedelen modtager både undervisning og lommepenge. Ud fra afsnit \ref{Okonomi}  og \ref{LommeStat}  kan man umiddelbart se, at børn modtager undervisning i emner relevant til økonomi omkring slutningen af andet forløb i folkeskolen. Derudover kan man se ud fra statistikkerne, at indføring af lommepenge først rigtig slår ud omkring 9 til 11 års alderen, hvor antallet sprang fra 38 procent til 78 procent. Derudover er det omkring den samme alder, at børn begynder at lære om økonomi i skolen. Et supplement til læring af økonomi vil derfor være mest nyttigt omkring den samme tid. Derfor vil løsningen være rettet mod denne aldersgruppe.

I afsnit \ref{LommeModeller} blev der undersøgt forskellige lommepengemodeller, samt fordele og ulemper ved disse modeller. Derudover bliver der i det senere afsnit \ref{ModelVurdering} yderligere evalueret på disse lommepengemodeller, som viste at især ‘Akkordmodellen’ og ‘Lommepenge et, pligter noget andet’ står ud som de bedste af modellerne. Ud fra dette vil der blive afgrænset til at der skal fokuseres på at implementere en eller muligvis flere modeller, baseret på de to førnævnte modeller. Helt præcis hvordan disse implantationer kommer til at se ud vil blive dækket yderligere i det kommende Design afsnit[[Ref når kapitel findes]], som skal prøve at finde frem til den bedste model, eller modeller, at implementere.

En potentiel løsning kan blive udviklet til mobiltelefoner, eller mere præcist til smartphones, som kan enten fungerer som platform for hele løsningen, eller som en applikation sådan at børn og forældre kan være forbundet til løsningen. Det kan nemlig ses at børn får deres første mobil når de er mellem otte og ti år gamle, og at det i seks ud af ti tilfælde er en Smartphone, som de får\cite{BornSmart}. Det er altså størstedelen af børn som har en Smartphone, så det vil derfor være oplagt at lave denne form for løsning. Situationen er dog at på grund af tidsbegrænsninger, og at løsningen ikke må blive lavet til en mobiltelefon. Der vil derfor ikke blive set på at udvikle en løsning til mobiltelefoner, og der vil derfor laves en løsning til en almindelig computer med windows eller linux.

Noget, der kunne være værd at kigge på i projektet, er, hvordan man kan lære børn og unge omkring økonomi på en passiv måde, sådan, at deres udvikling ikke bliver forstyrret. Dette skal forstås som, at børn skal have love til at være børn. En dreng på 9 år burde ikke behøve at bruge endeløse timer, for at få sin økonomi til at hænge sammen. Disse økonomiske erfaringer skulle gerne blive erfaret passivt, mens barnet får lov til at nyde sin barndom. Der er dog nogle problemer med at undersøge, og tage højde for, dette. Det er svært at vurdere om et barn har fået noget som helst ud af denne passive læring, da det vil tage flere år før man kan se om det har haft en effekt, når barnet er ældre og konsekvenserne er større. Det er ikke muligt at sige med sikkerhed, om et barn har fået noget ud af den passive læring, medmindre man har adskillige år til at undersøge om det havde en effekt. Rækkevidden og tiden for projektet er ikke stort nok, til at om den passive læring vil give den ønskede effekt på barnet, når det engang er blevet ældre. På grund af dette vil dette punkt blive afgrænset fra dette i projektet.

Til sidst er der afsnit \ref{Valuta}, som gør rede for fordelene og ulemperne, ved at basere en løsning på rigtige lommepenge, eller et fiktivt valutasystem. Begge sider af sagen har både mange positive og negative elementer til sig, så det er ikke helt klart hvad for en der vil fungere bedst i praksis, og give det bedst mulige svar på problemstillingen. Optimalt ville man prøve at implementerer begge muligheder, for at give flere muligheder til hvad brugeren kan med systemet. Dilemmaet her er at de to former er vidt forskellige, og skal derfor også implementeres og udvikles med vidt forskellige forhold i tankerne, som der ikke nødvendigvis er tid til. Derfor vil der umiddelbart blive afgrænset til et system baseret på rigtige lommepenge, som virker til at være bedre gearet til at lære om økonomi.
