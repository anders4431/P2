I dette kapitel er projektets tilhørende program beskrevet. Der er blevet gennemgået forskellige væsentlige elementer af programmet, som hvordan data i programmet kan persistere mellem sessioner, og hvordan data kan blive repræsenteret i grafer. Derudover er der også blevet gennemgået sikkerhed omkring brugerprofiler i programmet, diverse ændringer i forhold til hvad der blev fremlagt i kapitel \ref{Design}, samt testning for at sikre at programmet fungerer som forventet.

I starten af programudviklingen blev NDatabase brugt til at persistere data, da den var nem at implementere og god til at håndtere at klasser bliver ændret på. Desværre viste det sig at NDatebase var ret langsom og den blev skiftet ud med SQLite, der heldigvis var hurtigere. Programmet blev udviklet efter MVVM-designmønstret, som er velegnet til at lave et program med en brugergrænseflade. MVVM skulle også gerne gøre det lettere at teste programmet med unit-tests.
Det viste sig desværre at på grund af \textit{hardcoded} dependencies var dette ikke tilfældet. I fremtidige projekter skal programmet designes anderledes, så det bliver lettere at teste programmet. Eksempelvis efter \textit{dependency injektion}-designmønstret, som dog godt kan kombineres med MVVM. Der blev alligevel lavet nogle få unit-tests, som ikke afslørede fejl i koden. Resten af programmets funktioner blev testet ved at bruge programmet, og se om det virker. 

Vi kan  på baggrund af testene konkludere at programmet fungere efter de mest væsentlige punkter i produktkravene. De punkter der ikke blev overholdt kan læses om i afsnit \ref{VidereArbejde}.


