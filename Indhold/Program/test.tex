Til at teste programmet er der lavet en \textit{unit-test}. At unit-teste går ud på at dele koden op i så små testbare stykker som muligt og så teste hver del individuelt og på den måde sikre sig at hver lille del af programmet opfører sig som forventet. Derefter sætter man delene sammen og kan herefter teste interfacene mellem dem.
Unit testing har den fordel at man relativt hurtigt kan finde en eventuel fejl, da det er en lille del kode man tester ad gangen. 
Den gennemførte unit-test, tester login-funktionen. Herunder er der 4 test, der afprøver hvad login gør, hvis brugeren er et barn, hvis brugeren er admin, hvis brugernavn og adgangskode er forkert og hvis der slet ikke skrives noget brugernavn eller adgangskode. Under alle 4 scenarier opførte koden sig som forventet. Der burde optimalt set have været flere unit-test, men det viste sig at kildekoden til programmet ikke er særlig unit-test venligt. Derfor blev det kun til én unit-test og resten af koden er testet med en opdigtet familie, som oprettede pligter, påtog sig pligter og fik løn for dem. Derved blev programmets funktionaliteter testet, omend ikke så præcist og professionelt som hvis det var blevet gjort med unit-tests. 

