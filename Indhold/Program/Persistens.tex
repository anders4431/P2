I vores program har vi brug for, at kunne gemme data et sted så dataene kan hentes ind igen, når programmet starter. Dette kaldes persistens og kan opnås på flere forskellige måder, og vi har undersøgt nogle få nærmere.
De data vi gerne vil gemme til næste gang, programmet starter, er blandt andet oplysninger om, hvilke brugere der er i systemet, hvilke pligter der eksisterer og informationer om overførsler til de forskellige konti.

Til at løse denne opgave, har vi kigget på nogle oplagte muligheder, som herefter beskrives.

\subsubsection{XML fil}
XML er et simpelt format til at gemme information, der er let læseligt for både mennesker og computere. Det er eksempelvis informationerne der skal gemmes, struktureres med tags eller gemmes i en fil på harddisken\cite{xmlspecs}. Tekstfilen kan så åbnes med en teksteditor, og der kan nemt ændres i den og tilføjes data til filen.

%evt eksempel på noget xml værk?

\subsubsection{Relationel database}
En relationel database er god til at håndtere en stor mængde data med samme struktur. Data bliver delt op i tabeller med præcist definerede kolonner, og databasen selv understøtter en masse forskellige operationer (sorter efter, større end m.fl.), og kriterier for udvælgelse af hvilke datarækker der skal returneres. Her kiggede vi specifikt på SQLite, der er et lille bibliotek, der kan integreres direkte i programmet. Modsat dette findes dedikerede database systemer som MySQL og PostgreSQL, der kører som separate programmer.

Her ligger det store arbejde i koblingen mellem objekter i vores program og rækker i databasen. Alle variable i en klasse skal mappes over til kolonner i databasen og specielt referencer til andre objekter og egne datatyper, kan være svære at håndtere. Denne problemstilling kendes som "The object-relational impedance mismatch"\cite{ORIM}.

\subsubsection{Objektdatabase}
En objektdatabase er forskellig fra en relationel database på den måde, at den er baseret på objekter og derfor bare kan tage mod et objekt og gemme det.

Vi har valgt at gå videre med en object database af flere årsager. For det første er den er enkel at implementere. For det andet, vil slippe for, at skulle lave en "mapping" mellem klasser i C\# og kolonner i den relationelle database, hvor med en relationel database, vil "mappingen" mellem klasser og kolonner skulle opdateres eller omskrives hver gang en klasse ændres. Dette vil specielt være omfattende for os, da vi med vores begrænsede erfaring i objekt orienteret programmering, ofte vil lave store ændringer af vores klasser.
Til at klare opgaven som objekt database benyttes databasen NDatabase. NDatabase er en simpel open source object database lavet til C\#. Databasen understøtter LINQ, der er query sprog i Microsoft .NET Framework, der understøtter typesikkerhed og IntelliSense\cite{linqdok}.

Til at håndtere forbindelsen til databasen, er der lavet en statisk klasse. For at gemme fx et objekt af typen \code{transfer} i databasen, benytter vi "Store" metoden beskrevet i listing \ref{lst:db} . For så at hente transfers ud igen, benyttes metoden "GetTransfersForChild", der ved brug af en LINQ-forespørgelser returnerer overførslerne hørende til et objekt af typen "Child".
\\

\begin{lstlisting}[caption={Uddrag af filen "db.cs" fra kildekoden til programmet},label={lst:db}]
public static class db
{
	//En string dbFileName med filnavnet på databasen laves
	private const string dbFileName = "database.db";
	
	//Herefter åbnes database forbindelsen
	private static NDatabase.Api.IOdb odb = OdbFactory.Open(dbFileName);

	public static void Store(Transfer transfer)
	{
		//Objektet bliver sendt til databasen
		odb.Store(transfer);
		
		//Databasen bliver tvunget til at opdatere
		odb.Commit();
	}
	
	public static Collection<Transfer> GetTransfersForChild(Child child)
	{
		var transfers = from transfer in odb.QueryAndExecute<Transfer>()
						where transfer.Recipient != null && transfer.Recipient.FullName == child.FullName
						select transfer;

		Collection<Transfer> Transfers = new Collection<Transfer>();

		foreach (var t in transfers)
			Transfers.Add(t);

		return Transfers;
	}
}
\end{lstlisting}