Dette afsnit handler om de ændringer, der blev lavet til det aktuelle program, i forhold til hvad der blev beskrevet i det foregående Design afsnit. Her indgår nogle funktionelle ændringer, samt større ændringer til programmets layout.
Til at starte med var det meningen at det skulle være muligt for de administrerende brugere at ændre hvilken lommepengemodel, som børnene er underlagt. Dette er ikke blevet implementeret, og programmet gør kun brug af akkord modellen. Hændelsesforløbene som blev lavet viser at det var en mulighed for den almindelige bruger at fratage sig en pligt, efter den var blevet påtaget. I det aktuelle program blev dette dog ændret til at dette ikke var muligt, sådan at brugerne bliver nødt til at forpligte sig til pligten. Dette er ikke blevet implementeret. Se afsnit \ref{VidereArbejde}.
Problemområde-analysen beskrev en \code{Anmodning}-klasse i systemet, som blev sendt til en administrerende bruger når en almindelig bruger markerede en pligt til at være færdig. Denne klasse blev dog kasseret i det aktuelle program, og findes nu som en \code{Enum}-type, som beskriver hvilken tilstand en pligt er i. Dette blev gjort fordi \code{Anmodning}-klassen var unødvendig, og kunne lettere og mere effektivt blive gjort ved hjælp af en \code{Enum}-type, idet at \code{Anmodning}-klassen kun ville indeholde en reference til en \code{Pligt} og et \code{Barn}.