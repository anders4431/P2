For at programmet kan fungere optimalt, er det nødvendigt at brugeren logger ind i programmet. Herved får han hentet den data fra databasen, som er hans og han kan ikke ændre i andre brugeres data.
For at login fungere, er det nødvendigt at hver bruger har et unikt brugernavn og en adgangskode, som skal bruges under login. Brugernavnet gemmes i databasen og alt data hørende til brugeren gemmes sammen med brugernavnet. Hvis også adgangskoden blev gemt i databasen, ville det udgøre en sikkerhedsrisiko. Derfor gemmes der i stedet en hash-kode af adgangskoden. Hash-koden er en krypterede form af adgangskoden og er umulig at dekryptere igen, hvilket betyder, at hvis nogen hacker databasen, ville de ikke finde en liste med alle adgangskoder, men i stedet en hash-tabel.
Når en bruger logger ind i programmet, bliver brugernavnet slået op og programmet kan se hvordan hash-koden til adgangskoden burde se ud. Herefter laves en hash-kode af den indtastede adgangskode og det kontrolleres om den passer overens med hash-koden gemt i databasen. Hvis dette er tilfældet, er brugeres login godkendt og hans data hentes fra databasen og han kan gå i gang med at bruge programmet. Hvis hans login ikke godkendes, får han en fejlmeddelelse, der fortæller ham at enten brugernavn eller adgangskode var forkert. 
