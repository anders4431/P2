Ud fra den forudgående problemanalyse er det blevet vist, at børn og unge har en utilstrækkelig forståelse for økonomiske elementer som renter, lån og opsparinger, samt en manglende forståelse for økonomisk ansvar. Alt dette er stadig tilfældet, selvom størstedelen af børn og unge modtager lommepenge fra deres forældre, samt undervisning i økonomiske elementer i deres skole, som begge gerne skulle have hjulpet dem til at have viden om økonomi. Derudover er der også blevet undersøgt og vurderet de forskellige versioner af lommepengemodeller, som forældre bruger til at udlevere og administrere lommepenge. På baggrund af dette problem, og den forudgående analyse af problemet, er der blevet udarbejdet følgende problemformulering:

\begin{itemize}

	\item Hvordan kan man lave et system a la et netbanksystem, på en datalogisk, som er baseret på lommepenge og pligter?
	\item Hvordan kan man lære børn og unge omkring økonomi, samt give en bedre forståelse for økonomi, ved hjælp af en datalogisk løsning?

\end{itemize}

Ud fra dette vil der blive arbejdet hen mod en prototype, baseret på objektorienteret programmering, der kan bruges som et værktøj til at understøtte børn og unges læring af økonomi.