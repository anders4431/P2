Ud fra den forudgående problemanalyse er det blevet vist, at børn og unge har en utilstrækkelig forståelse for økonomiske elementer som renter, lån og opsparinger, samt en manglende forståelse for økonomisk ansvar og værdien af penge. Alt dette er stadig tilfældet, selvom størstedelen af børn og unge modtager lommepenge fra deres forældre, samt undervisning i økonomiske elementer fra deres skole, som begge gerne skulle have hjulpet dem til at opnå disse egenskaber. Derudover er der også blevet undersøgt og vurderet de forskellige versioner af lommepengemodeller, som forældre bruger til at udlevere og administrere lommepenge. På baggrund af dette problem, og den forudgående analyse af dette problem, er der blevet udarbejdet følgende problemformulering:

\begin{itemize}
\item Hvordan kan man, ved hjælp af en datalogisk løsning, give børn og unge en bedre forståelse for økonomi, baseret på et system omkring lommepenge og pligter?
\end{itemize}

Ud fra dette vil der blive arbejdet hen mod en prototype, baseret på objektorienteret programmering, der kan bruges som et værktøj til at understøtte børn og unges læring af økonomi.