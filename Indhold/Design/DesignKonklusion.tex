Dette kapitel kigger på problemområdet, og undersøger, forklarer og illustrerer det i form af objektorienteret analyse- og designmetoder, som beskrevet i \cite{ObjektAnalyseDesign}. Derudover har dette kapitel været omkring det grafiske delsystem WPF, og dertilhørende XAML, samt designmønstret MVVM. Til sidst blev der kigget på hvordan en mulig brugergrænseflade kan se ud til dette projekts løsning, hvilke overvejelser der har været angående det, og en gennemgang af hvad der blev valgt at primært arbejde videre med.

Analysen af problemområdet giver en ide til hvordan problemområdet, arbejdet omkring pligter i en husholdning, ser ud repræsenteret med klassediagrammer og hændelsesforløb. Dette giver en grundidé til hvordan klasserne i løsningen kan blive struktureret, samt de vigtigste funktionaliteter og hændelser som kan påvirke disse klasser i systemet. Disse observationer af problemområdet fungerer dog kun som en idé til hvordan en løsning kunne se ud, og kan derfor blive lavet anderledes i en aktuel løsning, men stadig beholde den beskrevne sammenhæng.

Til dette projekts løsning vil der blive gjort brug af MVVM designmønsteret kombineret med WPF og XAML. Dette blev valgt for at gøre testning af projektets løsning lettere, idet at logikken, som er det man vil teste, er separeret fra brugergrænsefladen, og testning behøver derfor ikke at skulle igennem, eller involverer, brugergrænsefladen. Dette designmønster gør det også lettere at implementere eventuelle udvidelser til systemet.

Brugergrænseflade afsnittet giver, ligesom problemanalyse afsnittet, en generel idé til hvordan en mulig brugergrænseflade kunne se ud. Derudover blev der også beskrevet en brugergrænseflade livscyklus, som kigger på hvordan brugergrænsefladen ændrer sig, idet en bruger åbner programmet, logger ind, osv.. Dette skulle gerne, sammen med klasserne og hændelserne beskrevet i problemområde analysen, give nogle idéer til mulige \textit{View}, \textit{ViewModel}, og \textit{Model} klasser til løsningen, baseret omkring MVVM designmønstret.