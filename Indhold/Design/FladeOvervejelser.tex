Det kunne muligvis være en god ide, at splitte ‘Pligter’ fanen for den administrerende bruger ud i to dele, da den indeholder en del funktioner, og kan let ende med at virke uoverskuelig. Det kunne deles sådan op, at den nuværende ‘Pligter’-fane, ville tage sig af ledige og påtagede pligter, hvor den administrerende bruger kan få overblik over pligterne samt tilføje og editere pligter, og en anden fane blev dedikeret kun til anmodninger fra de almindelige brugere.

En ‘Anmodninger’ fane ville indeholde en liste lignende den for ‘Pligter’, og give mulighed for at acceptere og afvise en markeret anmodning på listen. Dette vindue kunne også indeholde en anden liste, som tager sig af anmodninger angående hævninger fra de almindelige brugers konti. Dette kan dog også blive gjort som det før var blevet beskrevet, gennem ‘Børn’-fanen hvor et barn ville anmode en administrerende bruger om at hæve fra sin konto, og derefter få pengene udleveret. Den anden metode, hvor hævnings-anmodninger befandt sig i sin egen fane ville nok være bedst, hvis systemet og brugernes konto var opkoblet til en rigtig bankkonto, da det så ikke altid vil være muligt at udlevere pengene til brugeren med det samme. Med det ville et anmodnings system omkring hævninger ikke være nødvendigt, da brugerens konto er opkoblet til en bankkonto alligevel, som brugeren kan hæve fra. På grund af dette og at det ville gøre brugergrænsefladen og programmet yderligere kompliceret, som gerne skulle være brugbart for børn, så vil den originale metode og layout, se figur \ref{ForalderUI}, blive brugt.

En fane, som kan blive tilføjet til brugergrænsefladen, kan være en faner for indstillinger. Denne fane kunne give den almindelige bruger få funktioner, som ændring af brugernavn og kodeord. Den administrerende bruger kunne til gengæld få adskillige funktioner gennem sådan en fane, udover ændring af brugernavn osv. Denne fane kunne give mulighed for den administrerende bruger at definere en renteprocent, hvor mange pligter en enkelt bruger kan optage, default værdi for pligter osv., og med det give adskillige administrative muligheder. Dette ville dog betyde, at en almindelig bruger skulle abonnere på en af de administrerende brugere, for at få disse indstillingere defineret. Dette kan begrænse friheden af systemet, idet at en bruger med denne metode kun kan udføre pligter lavet af den administrerende bruger, som brugeren er abonneret til. På den anden side kunne man også gøre det således, at der kun kan findes en enkelt administrerende brugerprofil, som f. eks. et par forældre, der så deles om, at definerer indstillingerne for systemet, og samlet laver pligter og behandler anmodninger.
