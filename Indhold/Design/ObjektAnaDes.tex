Dette afsnit handler om, hvordan man kan beskrive projektets problemområde i form af klasser og hændelser. Metoderne brugt i dette afsnit er hentet fra bogen \textit{Objektorienteret analyse og design} \cite{ObjektAnalyseDesign}. Klasser er overordnede repræsentationer af objekter, som for eksempel en enkel \code{Pligt} eller et \code{Barn}. Hændelser er de begivenheder og ændringer, som klasserne kan forårsage eller blive påvirket af. Klasser kan normalt blive beskrevet som et navneord, for eksempel en \code{Admin} eller en \code{Pligt,} mens hændelser kan blive beskrevet som udsagnsord, for eksempel en \code{Admin} som opretter en ny \code{Pligt}, og et \code{Barn} som efterfølgende optager denne \code{Pligt}. Klasser og hændelser for dette problemområde er blevet sat op i en hændelsestabel, som kan blive set på tabel \ref{Haendelstabel}. En plads som har et plustegn betyder, at den tilsvarende hændelse kun forekommer en gang pr. klasse instans, og et gangetegn betyder, at det kan forekomme flere gange.

\begin{table}[htb]
	\small 
	\newcolumntype{C}{>{\centering}X}
	\setlength{\tabcolsep}{5pt}
	
	\begin{tabularx}{\textwidth}{|>{\bfseries}l|c|c|c|C|C|c|c|c|}
		\hline
		& \textbf{Pligt} & \textbf{Optage} & \textbf{Anmod} & \textbf{Behandle}
		& \textbf{Overføre} & \textbf{Ledige} & \textbf{Hæve} & \textbf{Stats} 	\tabularnewline \hline
		\textbf{\code{Admin}} & * &   &   & * & * & * & * & * 					\tabularnewline \hline
		\textbf{\code{Pligt}} & + & * &   &   & + &   &   &  					\tabularnewline \hline
		\textbf{\code{Barn}}  &   & * & * &   & * & * & * & * 					\tabularnewline \hline
		\textbf{\code{Model}} & * &   &   &   & * &   &   &  					\tabularnewline \hline
		\textbf{\code{Anmod}} &   &   & + & + &   &   &   &  					\tabularnewline \hline
	\end{tabularx}
	
	\caption{Hændelsestabel}
	\label{Haendelstabel}
\end{table} 
 
Tabellen giver et overordnet overblik til sammenhængen mellem klasser og hændelser, og være beskrivende for de mere væsentlige dele af problemområdet. En \code{Anmodning} er i denne sammenhæng en besked et barn sender til sin \code{Admin} om, at det har fuldført en af sine tilskrevne pligter og ønsker denne behandlet.

Klasserne kan herefter blive sat op i klasse diagram, som viser de indbyrdes sammenhænge mellem klasserne i problemområdet. Se figur \ref{KlasseDiagram}. Klassehierarkier bliver vist indkapslet i kasser, samt pile fra underklasser til superklasser. Rene streger viser associering, som \code{Model} og \code{Admin}, hvor lommepenge modellen går ind og påvirker på hvad forældrene gør i forhold til pligter og udbetalinger. En klasse som tilhører en anden klasse, bliver vist med en rombe i stedet for en pil. Ligeledes angiver tallene mængden af hver klasse til en anden klasse. For eksempel så er en \code{Pligt} altid kun forbundet til den \code{Admin} som lavede den, mens en \code{Admin} sagtens kan lave adskillige pligter.

\begin{figure}[H]
\centering
\includegraphics[width=0.8\textwidth]{Billeder/KlasseDiagram.png}
\caption{KlasseDiagram.}
\label{KlasseDiagram}
\end{figure}

Ud fra den foregående hændelsestabel samt klassediagram kan man beskrive klasser mere detaljeret ved hjælp af hændelsesforløb-diagrammer. Hændelsesforløb viser, hvordan et objekt af en klasse opstår i problemområdet og systemet, og hvad der sker med det indtil det ’dør’. Pile viser hændelser som hver klasse kan udføre, eller som kan blive udført på dem. Hændelser kan også forårsage, at et objekt skifter tilstand, som bliver vist med tilstandsbokse. Hændelser kan også føre tilbage til den samme tilstandsboks, som betyder at hændelsen kan blive udført adskillige gange. Det følgende viser fire hændelsesforløb for dette problemområde. Metode-hændelsesforløbet er blevet udeladt, som sagt fordi den påvirker andre områder i systemet og vil derfor ikke gøre andet end at være aktiv, indtil den bliver udskiftet.

\begin{figure}[H]
\centering
\includegraphics[width=0.8\textwidth]{Billeder/ForaelderForloeb.png}
\caption{\code{Admin}-hændelsesforløb.}
\label{ForaelderHaendelsesforloeb}
\end{figure}

\begin{figure}[H]
\centering
\includegraphics[width=0.8\textwidth]{Billeder/BoernForloeb.png}
\caption{\code{Barn}-hændelsesforløb.}
\label{BarnHaendelsesforloeb}
\end{figure}

\begin{figure}[H]
\centering
\includegraphics[width=0.8\textwidth]{Billeder/PligtForloeb.png}
\caption{\code{Pligt}-hændelsesforløb.}
\label{PligtHaendelsesforloeb}
\end{figure}

\begin{figure}[H]
\centering
\includegraphics[width=0.8\textwidth]{Billeder/AnmodningForloeb.png}
\caption{\code{Anmodning}-hændelsesforløb.}
\label{AnmodningHaendelsesforloeb}
\end{figure}
