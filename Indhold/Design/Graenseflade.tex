Dette afsnit handler om de overvejelser, der har været om brugergrænsefladen, til dette projekts løsning. Her indgår også et livscyklus-diagram, for hvordan brugergrænsefladen ser ud og ændrer sig, alt efter forskellige hændelser, såsom hvilken slags bruger der logger på. Den udgave af brugergrænsefladen, der blev valgt til at være grundmodellen for programmet, vil også blive gennemgået og illustreret.

Til at starte med vil brugergrænsefladen for programmet være et almindeligt login vindue. Efter en bruger har indtastet et gyldigt brugernavn og password, vil programmet begynde, at autentificere med en eventuel database. Alt efter hvilken bruger der har logget på, hentes den data fra databasen, som er tilknyttet den bruger som loggede på. Denne data vil herefter blive vist på brugerens grænseflade, som nu består af forskellige faner, knapper osv. Brugeren kan herefter interagere med denne data gennem brugergrænsefladen, indtil brugeren afslutter sessionen, hvor der til sidst vil blive synkroniseret med databasen, hvorefter programmet vil afslutte. Se på figur \ref{LifeCycle}.

\begin{figure}[H]
\centering
\includegraphics[width=1\textwidth]{Billeder/LifeCycle.png}
\caption{Brugergrænseflade/program-livscyklus}
\label{LifeCycle}
\end{figure}