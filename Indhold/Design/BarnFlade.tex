I Figur \ref{LifeCycle} kan man se at cyklussen udgrener sig efter at brugertypen er blevet fastlagt. En for administratorer, altså en forælder el. lign., og en for almindelige bruger, som for det meste ville være børn.

Ens for alle brugere af programmet er, at brugergrænsefladen er fanebaseret og har derfor forskellige sektioner af programmet delt ind i faner placeret øverst i brugerfladen. Begge former for brugere deler en række faner. ‘Startside’, ‘Pligter’ og ‘Statistik’, som ligner hinanden på mange punkter, men fungerer på forskellige måder.

‘Startside’-fanen for den almindelige bruger viser et nyhedsfeed, hvor brugeren kan se aktiviteten angående vedkommendes påtaget pligter, samt transaktioner fra eller til brugerens konto. Ved siden af kan brugeren se, hvor meget vedkommende har stående på sin konto. Se figur \ref{BarnUI} for at se dette og de andre faner.

Under ‘Pligter’-fanen kan den almindelige bruger se to lister. Den højre liste viser hvilke ledige pligter, der er tilgængelige for brugeren, som brugeren så kan påtage sig. Et element i denne liste vil vise forskellige oplysninger angående den pågældende pligt, som hvor meget den er værd, og dens eventuelle deadline. Hvis brugeren vælger at påtage sig en pligt, vil pligten blive overført til den venstre liste, som viser alle brugerens aktuelle pligter, og fjerner den herved fra listen af ledige pligter for alle andre almindelige brugere. Brugeren kan herefter sende en anmodning for en af de aktuelle pligter, som giver en administrator besked om, at pligten er færdig og klar til at blive evalueret.

'Statistik'-fanen viser 2 diagrammer, over brugerens aktivitet, samt en liste over saldo ændringer på brugerens konto. Det hele skal holdes forholdsvist simpelt, da den primære bruger af dette formentlig vil være børn.

\begin{figure}[H]
\centering
\includegraphics[width=0.9\textwidth]{Billeder/BarnUI.png}
\caption{Brugergrænseflade illustration for almindelige brugere.}
\label{BarnUI}
\end{figure}
