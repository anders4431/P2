% Slide Template
% Omslut slides med \begin{frame} og \end{frame}
% Standard LaTeX syntaks benyttes (itemize, tabular osv)
% Husk at tilføje eventuelle packages i headeren

\newcommand{\fordel}[1]{ % Item med en grøn '+' bullet
\begin{description}\item[\color{ForestGreen}\textbf{+}]{#1}\end{description}
}

\newcommand{\ulempe}[1]{ % Item med en rød '-' bullet
\begin{description}\item[\color{red}\textbf{--}]{#1}\end{description}
}

\newcommand{\headercell}[1]{ % Farvet celle med hvid tekst til tabeller
\cellcolor{aauprimary}\color{white}\small\textbf{{#1}}
}

\begin{frame}
\frametitle{Modeller}
\framesubtitle{\textbf{Model 1:} Det rene tilskud} 
    \begin{itemize}
        \item{Fast udbetaling}
            \begin{itemize}
                \item{En gang om ugen}
                \item{En gang om måneden}
            \end{itemize}
        \item{Ingen pligter} 
    \end{itemize}
    \vspace{\baselineskip}
    \pause
    \fordel{Barnet har stor frihed}
    \ulempe{Ikke arbejde for penge - ``Penge er en selvfølge''}
\end{frame}

\begin{frame}
\frametitle{Modeller}
\framesubtitle{\textbf{Model 2:} Lommepenge inkluderer pligter}
    \begin{itemize}
        \item{Faste pligter}
        \item{Fast udbetaling}
        \item{Pligter skal udføres for at få udbetalt}
    \end{itemize}
    \vspace{\baselineskip}
    \pause
    \fordel{Alle hjælper til i husholdningen}
    \ulempe{Gider ikke at arbejde udover faste pligter - giver ikke penge}
\end{frame}

\begin{frame}
\frametitle{Modeller}
\framesubtitle{\textbf{Model 3:} Akkordmodellen}
    \begin{itemize}
        \item{Udbetaling afhænger af udførelse af pligter}
        \item{Fast eller varierende beløb}
            \begin{itemize}
                \item{Større opgaver giver mere}
            \end{itemize}
    \end{itemize}
    \vspace{\baselineskip}
    \pause
    \fordel{Ligner mere virkelighedens økonomi (jobs m.m.)}
    \ulempe{Mulig mistillid til familiens forhold - ``kræver penge''}
\end{frame}

\begin{frame}
\frametitle{Modeller}
\framesubtitle{\textbf{Model 4:} Lomme penger et, pligter noget andet}
    \begin{itemize}
        \item{Fast udbetaling}
        \item{Forventes at pligter udføres}
            \begin{itemize}
                \item{Påvirker dog ikke udbetalingen}
            \end{itemize}
    \end{itemize}    
    \vspace{\baselineskip}
    \pause
    \fordel{Bedre sammenhold, man får ikke penge for alt}
    \ulempe{Svagt billede af økonomi - ingen sammenhæng mellem arbejde og løn}
\end{frame}

\begin{frame}
\frametitle{Modeller}
\framesubtitle{\textbf{Vurdering af modeller:} Kriterier}
    \begin{enumerate}
        \item{Økonomisk forståelse}
            \begin{itemize}
                \item{I hvor høj grad giver modellen en forståelse af virkelighedens økonomi?}
            \end{itemize}
        \pause
        \item{Husstandsbidrag}
            \begin{itemize}
                \item{I hvor høj grad bidrager modellen til sammenholdet i familien?}
            \end{itemize}
        \pause
        \item{Tidskrav}
            \begin{itemize}
                \item{Hvor meget tid kræves der af den unge at benytte denne model?}
            \end{itemize}
        \pause
        \item{Enkelthed}
            \begin{itemize}
                \item{Hvor kompliceret er modellen i f.eks. beregning af det udbetalte beløb?}
            \end{itemize}
        \pause
        \item{Digitaliserings relevans}
            \begin{itemize}
                \item{I hvor høj grad er det relevant at lave en digitalisering/systematisering af modellen?}
            \end{itemize}
    \end{enumerate}    
\end{frame}

\begin{frame}
\frametitle{Modeller}
\framesubtitle{\textbf{Vurdering af modeller:} Vurdering}
\begin{center}
    \begin{tabular}{| l | r | r | r | r |} 
    \hline
    \cellcolor{aauprimary} & \headercell{Model 1} & \headercell{Model 2} & \headercell{Model 3} & \headercell{Model 4} \\ \hline
    Økonomisk forståelse & 1 & 3 & 5 & 2 \\ \hline
    Husstandsbidrag & 1 & 3 & 3 & 4 \\ \hline
    Tidskrav (Højere er bedre) & 5 & 3 & 2 & 4 \\ \hline
    Enkelthed & 4 & 3 & 2 & 3 \\ \hline
    Digitaliserings relevans & 1 & 3 & 5 & 3 \\ \hline
    \headercell{I alt} & \headercell{12} & \headercell{15} & \headercell{17} & \headercell{16} \\ \hline
    \end{tabular}
\end{center}
\end{frame}