Formålet med denne analyse er at dokumentere P2 projektforløbet for gruppe A325. Derudover er meningen med analysen, at gruppen reflekterer over samarbejdet og arbejdsgangen gennem projektforløbet.
\\
Gruppen består af 6 personer som var sammen i P1-projektet samt én ny person. I P1-projektforløbet fungerede gruppen rigtig godt og derfor blev det besluttet at fortsætte sammen i P2 sammen med blot én ny person. Derved kendte størstedelen af gruppen hinanden og det gjorde at det var let at komme i gang med projektet, da  der ikke skulle bruges tid på at lære hinanden at kende. At en ny person kom ind i gruppen forvoldte ingen problemer, men gjorde tværtimod at der kom nogle nye ideér og meninger på banen. Forventning til projektforløbet var fra starten, at det skulle forløbe nogenlunde som det havde gjort i P1, at ambitionsniveauet skulle være højt og at projektet skulle ende med at ligge over middel. 
\\
Forventningen til udbyttet af projektet, var at gruppen ville lære mere om problemanalyse og det at skrive rapport. Derudover var alle opsat på at gøre brug af de erfaringer vi gjorde i P1 projektet og på den måde lave et bedre projekt. 
