\section{Beskrivelse}
I P1 udarbejdede vi i gruppen en gruppekontrakt. Denne fik vi dog ikke særlig meget glæde af, da ingen egentlig følte, at der var brug for den. Derfor blev der til P2 ikke lavet nogen kontrakt og i stedet aftalte vi mundtligt at der var ”mødepligt” til både forelæsninger, vejledermøder og gruppemøder. Hvilket vil sige, at hvis man ikke kunne komme, skulle der meldes afbud.

Til projektarbejdet i grupperummet havde vi ingen faste mødetidspunkter, men i stedet aftalte vi fra dag, til dag om der skulle mødes næste dag og i så fald hvornår. Hvis vi ikke skulle mødes dagen efter, aftalte vi så hvad hver især skulle lave derhjemme.
Facebook er blevet brugt i stor stil til indbyrdes kommunikation med hinanden, når vi sad hjemme og ikke var samlet i grupperummet. Her har vi eksempelvis delt forskellige hjemmesider mellem hinanden. 
Ellers har vi i grupperummet gjort rigtig meget brug af tavlerne, som har været et godt værktøj til at skabe overblik. Som eksempel har vi skrevet spørgsmål til vejleder op på tavlen, for ikke at glemme dem, og ligeledes ofte lavet en to-do liste over dagen. 

Engang imellem har vi i gruppen også husket at lægge arbejdet til side, og lave sociale ting sammen. Det har vi gjort for ikke at gå død i arbejdet og samtidigt styrke sammenholdet og samarbejdet i gruppen. Vi har eksempelvis haft ”spille-aften”, spist aftensmad sammen, gået i fredagsbar og i byen sammen. Derved sørger vi for at alt ikke bare drukner i fagligt arbejde, men at vi også kan have det sjovt sammen.

Kommunikationen i gruppen har fungeret godt, ingen er blevet uvenner eller overhørt. Alle er af den mening at alle i gruppen siger deres mening og at alle hører efter når nogen taler. Derved gives der plads til alle i gruppen, og  samarbejdet har fungeret optimalt gennem hele projektforløbet.
\\
\\

\section{Analyse}

\emph{Gode erfaringer:}
\begin{itemize}

\item	Det er fint ikke at have en gruppekontrakt, da gruppen kendte hinanden i forvejen.

\item	Det fungerer godt at man ikke nødvendigvis skal møde hver dag, hvis der alligevel ikke er forelæsning. Nogen i gruppen synes bedre om at arbejde hjemme.  

\item	Godt at alles meninger bliver respekteret og hørt.

\item	Sociale arrangementer udenfor arbejdstiden er gode og vigtige at holde fast i.

\item	 Brugen af tavlerne i grupperummet skal vi fortsætte med, da det gør at alle er med på hvad der foregår. 
\end{itemize}\emph{Dårlige erfaringer:}
\begin{itemize}

\item	 Nogle gange er folk meget dårlige til at give svar på forespørgsler på Facebook. Dette var frustrerende for den, som stillede spørgsmålet.

\item	Mødetiderne blev ikke altid overholdt, hvilket gav frustration for den del af gruppen der var mødt til tiden.
\end{itemize}

\section{Forbedringer til P3}
Det er en god idé med gruppekontrakt, hvor man kunne aftale faste mødetider, så dette ikke skal aftales over Facebook. Derudover bør der være en konsekvens hvis man ikke kommer til tiden.
