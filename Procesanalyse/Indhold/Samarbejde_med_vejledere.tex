\section{Beskrivelse}
Samarbejdet med vores vejleder har været godt. Vi valgte sammen med vejlederen, at der ikke var behov en decideret kontrakt, men at vi bare ville tage det som det kom. Vi aftalte at gruppen skulle skrive dagsorden inden hvert møde og ligeledes skrive referat under hvert møde. Derudover var det primært vejleder der bestemte hvornår vi skulle holde møder, da han havde lang transporttid og 3 andre grupper at koordinere med. Inden vi havde første møde med vejleder gjorde vi os nogle få overvejelser om hvordan vi ville bruge ham:
\begin{itemize}
\item	Efter behov ville vi gerne sende henholdsvis rapport og kode, til gennemlæsning af vejleder, med håb om efterfølgende feedback.

\item	Vi vil gerne have regelmæssige møder, med henblik på hele tiden at holde os på rette spor med hjælp fra vejleder.

\item Til sidst ville vi gerne have mulighed for løbende at sende spørgsmål over mail, når de skulle melde sig.
\end{itemize}
Vejleder var enig i måden at bruge ham på og vi blev enige om at samarbejdet skulle fungere sådan. Det har kørt sådan at vejleder sendte en mail til os om, hvornår vi skulle holde møde. Mindst en dag inden mødet sendte vi en mail til vejleder, med den nyeste rapport og eventuelle spørgsmål. Til mødet talte vi så om hvad status på projektet var. 

En i gruppen blev valgt til referent og var således referent til hvert møde, så vi altid fik skrevet referat. Vi mener selv, at vi i dette projekt har fået rettet op på de dårlige erfaringer vi gjorde i P1. En dårlig erfaring var eksempelvis at vi fik holdt alt for få vejledermøder, hvilket gjorde at vi nogle gange arbejdede ud af et forkert spor, eller simpelthen fik lavet for lidt. 

\section{Analyse}

\emph{Gode erfaringer:}
\begin{itemize}

\item Godt med ugentlige vejledermøder. På den måde fik vi hele tiden holdt status over projektet og kom hele tiden videre.

\item Godt, at den samme person var referent hver gang, hvilket gjorde at referatet altid var opbygget på samme måde og gav et fint overblik.

\item Det fungerede fint at vejleder havde adgang til vores buildserver, så han altid havde adgang til den nyeste version af rapporten. \ref{projektplan}

\item Det fungerede nogenlunde at vejleder havde adgang til vores kode via GitHub.

\end{itemize}
\emph{Dårlige erfaringer:}

\item Der er et potientielt problem med at vejleder kan se koden hele tiden, da denne ikke altid fungerer optimalt.

\begin{itemize}
\item	Til tider blev dagsordenen ikke lavet forud for mødet, hvilket betød at gruppen ikke var forberedt tilstrækkeligt til mødet.

\end{itemize}

\section{Forbedringer til P3}

I næste projektskal vi være bedre til at udarbejde en dagsorden ordentligt inden hver møde, så vi på den måde sørger for hverken at spilde vores eller vejlederens tid. Der skal holdes fast i at holde ugentlige vejledermøder, da dette fungerede rigtig godt. 
