\section{Beskrivelse}
Samarbejdet med vores vejleder har været godt. Vi valgte sammen med vejlederen at der ikke var behov en decideret kontrakt, men at vi bare ville tage det som det kom. Vi aftalte at gruppen skulle skrive dagsorden inden hvert møde og ligeledes skrive referat under hvert møde. Derudover var det primært vejleder der bestemte hvornår vi skulle holde møder, da han havde lang transporttid og 3 andre grupper at koordinere med. Inden vi havde første møde med vejleder gjorde vi os nogle få overvejelser om hvordan vi ville bruge ham:
\begin{itemize}
\item	Efter behov ville vi gerne sende henholdsvis rapport og kode, til gennemlæsning af vejleder, med håb om efterfølgende feedback.

\item	Vi vil gerne have regelmæssige møder, med henblik på hele tiden at holde os på rette spor med hjælp fra vejleder.

\item	Sidst ville vi gerne have mulighed for løbende at sende spørgsmål over mail, når de skulle melde sig.
\end{itemize}
Vejleder var enig i måden at bruge ham på og vi blev enige om at samarbejdet skulle fungere sådan. Det har kørt sådan at vejleder sendte en mail til os, om hvornår vi skulle holde møde. Mindste en dag inden mødte sendte vi så en mail til ham, med den nyeste rapport og eventuelle spørgsmål. Tiil mødet talte vi så om hvad status på projektet var. 

En i gruppen blev valgt til referent og var således referent til hvert møde, så vi altid fik skrevet referat. Vi mener selv vi i dette projekt har fået rettet op på de dårlige erfaringer vi gjorde i P1. En dårlige erfaring var eksempelvis at vi fik holdt alt for få vejledermøder, hvilket gjorde at vi nogen gange arbejde ud af et forkert spor, eller simpelthen fik lavet for lidt. 

\section{Analyse}

\emph{Gode erfaringer:}
\begin{itemize}

\item Godt med ugentlige vejledermøder. På den måde fik vi hele tiden holdt status over projektet og kom hele tiden videre.

\item At den samme person var referent hver gang, gjorde at referatet altid var opbygget på samme måde og det gav et fint overblik.
\end{itemize}\emph{Dårlige erfaringer:}
\begin{itemize}
\item	Der var som oftest sat en hel time af til vejledermøderne. Ofte var vi færdige med det projekt-relevante stof indenfor en halv time, hvorefter resten af mødet gik med snak om alt muligt andet. Nogen i gruppen synes det var hyggeligt med den "ligegyldige" snak, mens andre i gruppen synes det var tidsspild.

\end{itemize}

\section{Forbedringer til P3}

I næste projekt skal vi sørge for at tale med vejleder om hvor lange vi mener møderne bør være, således at gruppen ikke føler at tiden spildes. Der skal holdes fast i at få holde ugentlige vejldermøder, da dette fungerede rigtig godt. 
