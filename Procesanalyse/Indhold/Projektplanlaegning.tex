\section{Beskrivelse}
Som følge af erfaring fra P1-projektet, skulle vi i P2 følge en tidplan for projektperioden. Derved blev der i starten af projektet, ved hjælp af programmet \textit{Gantter}, udviklet en tidsplan for projektet, hvori de forskellige deadlines for projektet blev lagt ind i. Men ligesom på P1-projektet lykkedes det desværre heller ikke denne gang at hverken følge tidsplanen eller holde denne opdateret. I stedet delte vi blot opgaver ud fra dag til dag og aftalt deadlines samtidigt. Når vi så mødtes for at arbejde i grupperummet opgjorde vi status af projektet og aftalte hvornår vi skulle mødes næste dag. 

Som følge af erfaring fra P1 har vi til versionsstyring mellem gruppens medlemmer brugt GitHub og det virkede atter rigtig godt. Samtidigt har vi brugt \LaTeX{} som formateringssporg til at skrive rapport i.   

Ligesom i P1 havde vi heller ikke i dette projektforløb en projektleder. Grunden er at alle er enige om at i P1 fungerede det godt at der ikke er én person der har styringen, med at alle istedet har ansvaret for at holde arbejdet i gang. Ingen andre grupperoller blev delt ud.  

Lidt over en måned inde i projektforløbet, blev der afholdt statusseminar. Her blev problemanalysen for projeket og vores ideér til det videre arbejde fremlagt for en opponentgruppe, samt vejledere. Derefter fik vi kommentarer og nye overvejelser med hjem. Derudover så vi fremlæggelse fra en anden gruppe og kunne derved hente lidt inspiration fra dem, til brug i vores eget projekt. Efter statusseminaret rettede vi selvfølgelig det, som vi havde fundet ud af kunne blive bedre, og derefter gik vi lidt i stå. GitHub.com danner en graf over hvor meget aktivitet, der har været i projektet i forhold til tiden, se bilag \ref{bilag2}. Denne graf ligner meget den tilsvarende graf fra P1, på trods af at vi efter P1 ønskede at ændre denne arbejdsaktivitet. Grafen på bilag \ref{bilag2} viser, at vi kom lidt langsomt fra start, men at fik lavet rigtig meget inden statusseminaret. Efter seminaret er aktiviteten kraftigt dalende hen mod midten af april hvor aktiviten nærmest står stille i den uge hvor vi skulle lave eksamensprojekt i OOP. Hen imod aflevering af projektet stiger aktiviteten kraftigt. 

\section{Analyse}

\emph{Gode erfaringer:}
\begin {itemize}
\item  GitHub er fortsat glimrende til versionsstyring.

\item	Godt at ansvar for styring af gruppen ikke placeres på én projektleder. 

\item	Statusseminaret var godt. Gruppen øvede sig i at fremlægge projekt og der kom nye idéer på banen.
\end{itemize}\emph{Dårlige erfaringer:}
\begin{itemize}
\item	Også i dette projekt så vi en skiftende arbejdsindsats, som først kom i top hen mod en deadline (statusseminar og endelig aflevering). Det havde været bedre med en konstant arbejdsindsats. 

\item	Den manglende styr på tidplanen, gjorde at vi indimellem mistede overblikket over projeket. 

\item	Ikke at dele nogle roller ud i gruppen. En rolle kunne være at holde styr på tidsplanen.
\end{itemize}

\section{Forbedringer til P3}

I næste projekt kunne det være en god idé at forsøge med uddelling af roller mellem gruppens medlemmer. Eksempelvis kunne man have en sekretær, som kunne sikre at der bliver styr på tidplanen, samt holde kontakt til vejleder osv. Man kunne derudover vælge en person til at være ordstyrer og derved sikre at diskussioner i gruppen, foregår på en ordentlig måde. Man kunne eventuelt skiftes til at påtage sig de forskellige roller. 

Ved at vælge én person til at holde styr på tidsplanen, kunne man måske sikre at dette rent faktisk bliver gjort. Det kunne være godt at undgå at miste overblikket, som det ind imellem er sket i denne projektperiode. Derved kunne man også få flere deadlines på tidplanen, således at arbejdsindsatsen ikke bliver så ”bølget”.

