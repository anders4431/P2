\section{Beskrivelse}
Ved siden af projektarbejdet, har alle i gruppen deltaget i 3 kurser; Diskret Matematik (DMAT), Computer Arkitektur (CART) og Objekt Orienteret Programmering (OOP). Med hensyn til projektet har vi klart fået mest ud af OOP. Dette kursus handlede naturligvis om programmering og sproget der blev brugt var C\#. C\# var netop det sprog som programmet, der skulle udvikles i forbindelse med P2, skulle skrives i. Derved lærte vi det, som var nødvendigt for at vi kunne udleve kravene til P2. DMAT og CART er kurser, som er relevante for vores udannelse generelt og i senere projekter kommer vi til at kunne gøre godt brug af dem. Men til dette projekt har vi ikke rigtigt kunnet bruge hverken DMAT eller CART.
I forbindelse med alle kurserne sad gruppen sammen og løste kursernes opgaver i fællesskab. Det er dog lige med undtagelse af CART, hvor nogen i gruppen havde lidt svært ved at fastholde koncentrationen og interessen.
Med hensyn til selve projektarbejdet, delte vi arbejdet ud mellem gruppens medlemmer, på en såden måde at alle lavede lidt af det hele. I P1 delte gruppen sig på et tidspunkt i to således at den ene del arbejdede videre på rapporten og den anden del arbejdede på programmet. Dette resulterede i at det kun var den ene halvdel af gruppen der var godt inde i programmet. Det har vi forsøgt at ændre i P2.
Derudover tog vi en god erfaring med fra P1, hvilket er, at når et afsnit til rapporten er blevet færdiggjort, skal mindst to andre læse det igennem. På den måde får man rettet afsnittene løbende og alle lærer om hele projektet og ikke bare den del man selv har skrevet. 

\section{Analyse}

\emph{Gode erfaringer:}
\begin{itemize}
\item Det er godt at gruppen hjælper hinanden med kursus-opgaver. Det giver godt sammenhold og sammarbejde i gruppen. 

\item Det er godt at alle arbejder på alle sider af projektet, så alle er sat godt ind i det hele.

\item	Fortsat virker det godt, at mindst 2 personer læser og giver kommentarer til nye afsnit, inden disse bliver tilføjet rapporten.
\end{itemize}\emph{Dårlige erfaringer:}
\begin{itemize}
\item	Gruppen burde måske have forsøgt at få alle med i CART, så der ikke var nogen der kom bagefter.

\end{itemize}	 

\section{Forbedringer til P3}
Alle i gruppen skal selv tage et ansvar for at deltage aktivt i kurserne, så dette ikke kommer til at påvirke gruppen negativt senere hen. Derudover skal vi fortsat have fokus på at alle løbende forstår hele projektet og alle derved lærer det samme. Dette skal sikre at gruppemedlemmerne ved alt om hele projektet og ikke mest om den del af projektet de selv har skrevet. 
